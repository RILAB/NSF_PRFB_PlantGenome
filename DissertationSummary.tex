%setcounter{page}{1}
\renewcommand{\thepage}{Dissertation Summary}
\required{Dissertation Summary - Kimberly J. Gilbert}
%\begin{center}
%%\emph{Maximum of 15 pages}
%\end{center}


%\section*{Dissertation Summary}
~~~~A major obstacle in evolutionary biology is the difficulty of population genetic inference in the face of confounding factors, such as demographic history. My dissertation work has focused on several topics related to this broad area of research:
\begin{enumerate}[nolistsep]
\item Evaluating the ability of statistical genetic methods to estimate effective population sizes in the face of migration \citep{Gilbert:2015io}
\item Assessing the factors related to local adaptation at range edges during species expansion
\item Validating SNP loci under selection for adaptation to climate in lodgepole pine (\emph{Pinus contorta})
\end{enumerate}

	Effective population size, $N_e$, is a fundamental parameter in population genetics, evolutionary biology, and conservation biology, yet its estimation can be fraught with difficulties. Several methods to estimate Ne from genetic data have been developed that take advantage of various approaches for inferring $N_e$. The ability of these methods to accurately estimate $N_e$, however, has not been comprehensively examined. This part of my dissertation work employed seven of the most cited methods for estimating $N_e$ from genetic data (Colony2, CoNe, Estim, MLNe, ONeSAMP, TMVP, and NeEstimator including LDNe) across simulated datasets with populations experiencing migration or no migration. The simulated population demographies were an isolated population with no immigration, an island model metapopulation with a sink population receiving immigrants, and an isolation by distance stepping stone model of populations. We found considerable variance in performance of these methods, both within and across demographic scenarios, with some methods performing very poorly. The most accurate estimates of $N_e$ can be obtained by using LDNe, MLNe, or TMVP; however each of these approaches is outperformed by another in a differing demographic scenario. Knowledge of the approximate demography of population as well as the availability of temporal data largely improves $N_e$ estimates.
	
%	While a demographic history of ongoing gene flow can confound estimates of $N_e$, as described above, other types of demographic history can likewise cause difficulty in estimating parameters or understanding the processes and patterns contributing to the genetic makeup of individuals. The remaining two topics of my dissertation research explore the process of species range expansions through simulations, where predictions of genetic processes change beyond expectations due to an interaction of spatial movement and population bottlenecking, and use empirical data to validate methods for identifying loci under adaptation in the face of a complex demographic history in the lodgepole pine (\emph{Pinus contorta}).	% doesn't quite tie things together right
	
	Species range edges have boundaries that cannot always be explained ecologically or geographically, which leaves the question of what evolutionary forces may prevent populations at range edges from adapting and expanding the species range further. A large body of theoretical work has investigated many evolutionary parameters' effects on local adaptation in edge populations, but one area lacking in research is that of the interaction of the landscape with the ability to locally adapt. This study investigates how more realistic, heterogeneous environmental gradients (compared to the linear gradients that previous studies investigate) may interact with dispersal distance and the effect size of mutations. I have simulated a range of parameter combinations that show a strong relation of mutation effect size on the ability to spread across the landscape. As environmental heterogeneity increases, migration load (reduction in fitness due to dispersal away from an area previously adapted to) increases, and local adaptation becomes more difficult, especially in smaller populations at the range edge, slowing the speed of expansion across the landscape.
	
	A history of range expansion can confound many inferences that population genetics aims to understand. Identifying the loci that underlie traits contributing to local adaptation is one such inference that is a major goal in evolutionary biology today. The lodgepole pine (\emph{Pinus contorta}) is a major timber tree in the Pacific Northwest which has a history of expansion post-glaciation, and either one or putatively a second glacial refugia from which this expansion occurre1.9d. Climate change is spurring foresters to plant trees for future harvest that will be best adapted to future climates for optimal yield, hence identifying loci underlying adaptation to climate change is a key goal. I am conducting a validation study of SNP loci identified through GWAS, genotype-environment association, and $F_{ST}$ outlier tests to assess how often these methods may produce false positives as a result of population structure and spatial autocorrelation of genetic clines due to range expansion with gradients in environmental variables (i.e. temperature and precipitation). I have sampled a provenance trial (common garden study) in British Columbia to compare performance of populations from a range of native temperatures (MAT -3.7$^{\circ}$C - 11$^{\circ}$C) planted across test sites of varying temperature (MAT -1.4$^{\circ}$C - 5$^{\circ}$C) from which I will be able to test if predicted alleles do indeed show increased performance in mature, natural-grown trees.
	


