%\required{Project Summary} % remove the title to save some space

% Prepare Project Summary (limited to one page) of Proposed Fellowship Activities, including both research and training. Select GO next to Project Summary. This is an abstract of the proposed research and training. You must clearly address and identify in separate statements using the three boxes: (1) an overview of your proposed fellowship activities; (2) intellectual merit; and (3) broader impacts of the activities. Without these 3 sections, your application will be returned without review. List your sponsoring scientist(s) and institution(s) in the overview. Upload text in the proper boxes. Do not use the Supplementary Document option for the summary. Do not use jargon and abbreviations in the summary. It should be understandable by scientists not in your specialized field


% WHOLE DOC MUST BE 1 PAGE OR LESS
\paragraph{Overview} 

<<<<<<< HEAD
Adaptation to new and changing conditions will be vital in the face of climate and other anthropogenic changes. Understanding and predicting a species's ability is thus key, and is especially relevant in crops needed to sustain food and fuel sources for the world's growing populations.  Most ecologically or economically important phenotypic traits have a complex, quantitative genetic basis, yet we know relatively little about how quantitative trait evolution affects the genes underlying such phenotypes. \textbf{This research project aims to build models to understand and predict how phenotypic evolution affects, and is affected by, the underlying genetic architecture of quantitative traits.}\jri{bold too much?}  
Taking advantage of the genotypic and phenotypic resources available for maize and its wild relative teosinte, I propose to: 1) build a simulation model that accurately recapitulates the number and effect size of genes underlying important phenotypic traits in teosinte; 2) use this model to predict the impact of demography and selection on traits in domesticated maize; 3) test predictions of the model in existing teosinte-maize mapping populations; and 4) apply the model to predict the genetic architecture impacts, and is impacted by, complex demography and local adaptation.
%  is the distribution of effect sizes for new mutations in genomic regions underlying important phenotypic traits in maize and teosinte, and 2) how do different histories of demography and selection change this genetic architecture over time and impact the species' ability to adapt? Vast n which to investigate these questions. Through four objectives I aim to answer these questions using both real and simulated data to create and test theoretical predictions in the system. I will first estimate the distribution of fitness effects of new mutations in teosinte, maize's progenitor, then use this information to parameterize and simulate the demographic and selective history of maize during its domestication. This creates expectations for how the genetic architecture of various traits, some of which are hypothesized to be under selection during domestication, may change and can be compared to real data to test if predictions match reality. These results can be validated using a synthetic \emph{Zea} line what contains both teosinte and maize in the same background. Further, to investigate a broader range of demographic and selective pressures, I will simulate the history of various landraces of maize that expanded across the Americas to assess further impacts of these forces on genetic architecture of important traits. 
This research will be conducted at the University of California, Davis, under the supervision of Dr. Jeffrey Ross\--Ibarra and co-supervised by Dr. Kevin Thornton at the University of California, Irvine.
=======
The ability of organisms to adapt to new and changing conditions is vital in the face of climate or other anthropogenic changes. Understanding and predicting this ability to adapt for species is thus key, and is especially relevant in crop species that are needed to sustain food and fuel sources for the world's growing populations. This research project aims to examine the genetic architecture of important traits for adaptation in maize before and after its domestication bottleneck. Two major questions will be addressed: 1) what is the distribution of effect sizes for new mutations in genomic regions underlying important phenotypic traits in maize and teosinte, and 2) how do different histories of demography and selection change this genetic architecture over time and impact the species' ability to adapt? Vast resources of genomic and phenotypic data in both teosinte and maize create an ideal system in which to investigate these questions. Through four objectives I aim to answer these questions using both real and simulated data to create and test theoretical predictions in the system. I will first estimate the distribution of fitness effects of new mutations in teosinte, maize's progenitor, then use this information to parameterize and simulate the demographic and selective history of maize during its domestication. This creates expectations for how the genetic architecture of various traits, some of which are hypothesized to be under selection during domestication, may change and can be compared to real data to test if predictions match reality. These results can be validated using a synthetic \emph{Zea} line what contains both teosinte and maize in the same background. Further, to investigate a broader range of demographic and selective pressures, I will simulate the history of various landraces of maize that expanded across the Americas to assess further impacts of these forces on genetic architecture of important traits. This research will be conducted at the University of California, Davis, under the supervision of Dr. Jeffrey Ross\--Ibarra and co-supervised by Dr. Kevin Thornton at the University of California, Irvine.
>>>>>>> fb86488e9c1a1befb3d6fd904caee668f6450c1e

\vspace{-0.4cm}

\paragraph{Intellectual Merit}  

%\jri{this paragraph great for intellectual merit} These hypothesized changes in the DFE are not well understood in any system and are controversial in humans (cite lohmueller vs. pritchard etc.). Deleterious alleles likely play a large role in many adaptive phenotypes: crop plants have undergone dramatic demographic shifts, usually involving a domestication bottleneck followed by expansion as cultivation spread, and some authors even argue that selection on domestication traits has inadvertently increased the frequency of alleles deleterious for other phenotypes (cite gunther2010). Consistent with this, it has recently been shown that genes associated with a number of quantitative traits in maize are enriched for deleterious alleles compared to randomly chosen genes (cite mezmouk2014). Such information is crucial for understanding variation in phenotype, designing breeding strategies, utilizing diversity from wild relatives, or even engineering new traits using biotechnology. 

The proposed research will greatly expand our understanding of the adaptability of quantitative traits. Although most traits are  quantitative, we know little about how the genes underlying such traits --- their genetic architecture --- limits and responds to evolutionary change. Does adaptation use new mutations or standing variation? Are adaptive traits primarily driven by few loci of large effect or many loci of small effect? Does population fluctuation change the architecture of quantitative traits? Statistical and sample size limitations have meant that traditional QTL and GWAS approaches have been unable to answer these questions. My simulation based approach, calibrated using empirical data from maize and teosinte, has the potential to answer these and many other related questions about adaptation of quantitative traits.  Maize and teosinte  serve as a perfect system for this study as the evolutionary history (domestication) is relatively well known and there are extensive genomic and phenotypic data for both ancestral (teosinte) and derived (maize) populations. Furthermore, this work will advance our understanding of plant biology and plant genomics and inform breeding efforts in maize and other species to adapting to future climate change. 
%The results could inform crop studies in terms of maintaining genetic diversity for the future in ways that may have much larger long term impacts on the maintenance of diversity as well as on adaptation to changing environments.


\vspace{-0.4cm}

\paragraph{Broader Impacts}

The impacts of this research will span both the scientific community and the broader public. The research will be made publicly available at all stages through online repositories of simulation and analysis code as well by public archiving the data. Additionally, I will make published results available on public pre-print servers, or when possible in open-access journals. I have a strong record of conference attendance that I will maintain in order to present results to the scientific community, and will also use my presence on Twitter and as a contributor to The Molecular Ecologist online blog to more frequently reach both the scientific and public audiences. Furthermore, I will mentor undergraduate students in the lab, teaching skills on computational and genomics analyses and develop workshops on population and quantitative genetics and computational tools that I will organize to teach for graduate students at the Langebio research center in Irapuato, Mexico that is home to a number of collaborators to the Ross-Ibarra lab.  I will also have the opportunity to develop undergraduate teaching modules for a `flipped' version of the genetics class taught by Dr. Ross-Ibarra.


%and also be able to participate in the educational partnership that has been created between UC Davis and Pioneer High School in Woodland, California. This will allow me to visit high school classrooms to teach about the cutting edge of genomics and evolutionary biology research to young scientists as well as to potentially mentor high school students in the lab or on small-scale projects. 

%\kjg{found this here: \url{http://www.dailydemocrat.com/general-news/20140307/uc-davis-partnership-with-pioneer-high-benefits-budding-scientists}}
%\jri{didnt know about this. would have to get buy-in from brady/sinha. possible if you're interested, but if students want molecular bnechwork experience doesn't fit with current proposal...}. \jri{workshop on maize quant gen would be cool. have collaborators at Langebio in Irapuato where you could present. also could design a youtube video on it? another possibility -- i teach undergraduate genetics, and am in process of flipping classroom. you could help design modules (blog posts, videos, etc.) on pop and quant gen for that class, which would potentially could used for all the genetics students at UCD}.
% kind of just spitballing right now, need to make sure these things make sense/would work




