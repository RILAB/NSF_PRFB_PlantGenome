%\required{Project Summary} % remove the title to save some space

% Prepare Project Summary (limited to one page) of Proposed Fellowship Activities, including both research and training. Select GO next to Project Summary. This is an abstract of the proposed research and training. You must clearly address and identify in separate statements using the three boxes: (1) an overview of your proposed fellowship activities; (2) intellectual merit; and (3) broader impacts of the activities. Without these 3 sections, your application will be returned without review. List your sponsoring scientist(s) and institution(s) in the overview. Upload text in the proper boxes. Do not use the Supplementary Document option for the summary. Do not use jargon and abbreviations in the summary. It should be understandable by scientists not in your specialized field


% WHOLE DOC MUST BE 1 PAGE OR LESS
\paragraph{Overview} 

The ability of organisms to adapt to new and changing conditions is vital in today's world of climate and other anthropogenic changes. Understanding and predicting this ability to adapt for species is thus key, and is especially relevant in crop species that are needed to sustain food and fuel sources for the world's growing populations. This research project aims to examine the genetic architecture of important traits for adaptation in maize before and after its domestication bottleneck. Two major questions will be addressed: 1) what is the distribution of effect sizes for mutations % reword? mutations don't underly the trait, loci do
underlying important phenotypic traits in maize and teosinte, and 2) how do different histories of demography and selection change this genetic architecture over time and impact the species' ability to adapt? Vast resources of genomic and phenotypic data in both teosinte and maize create an ideal system in which to investigate these questions. I propose three objectives to answer these questions using both real and simulated data to create and test theoretical predictions in the system. The first objective is to estimate the distribution of fitness effects of new mutations in teosinte, maize's progenitor. From this result, simulations in objective two will be parameterized and run to simulate % used simulate twice in same sentence, fix
the demographic and selective history of maize during its domestication. This creates an expectation for how the genetic architecture of traits in maize, hypothesized to be either under selection during the domestication event or not, should change given its past and can be compared to real data to see if predictions match reality. Objective three then performs further simulations of additional demographic and selective events that varying landraces of maize have experienced to assess if further changes to genetic architecture are detected. This research will be conducted at the University of California, Davis, under the supervision of Dr. Jeffrey Ross\--Ibarra and cosupervised by Dr. Kevin Thornton at the University of California, Irvine.

\paragraph{Intellectual Merit}  

%\jri{this paragraph great for intellectual merit} These hypothesized changes in the DFE are not well understood in any system and are controversial in humans (cite lohmueller vs. pritchard etc.). Deleterious alleles likely play a large role in many adaptive phenotypes: crop plants have undergone dramatic demographic shifts, usually involving a domestication bottleneck followed by expansion as cultivation spread, and some authors even argue that selection on domestication traits has inadvertently increased the frequency of alleles deleterious for other phenotypes (cite gunther2010). Consistent with this, it has recently been shown that genes associated with a number of quantitative traits in maize are enriched for deleterious alleles compared to randomly chosen genes (cite mezmouk2014). Such information is crucial for understanding variation in phenotype, designing breeding strategies, utilizing diversity from wild relatives, or even engineering new traits using biotechnology. 


The proposed research will greatly expand our understanding of the adaptability of quantitative traits and the relationship this has with the architecture underlying such quantitative traits. Little is known about the genetic architecture of quantitative traits, yet most important traits are quantitative. With the availability of genomic and phenotypic data, this is an area now ripe for further research %I don't really like this sentence
A standing question in evolutionary biology is whether adaptation generally happens from few loci of large effect or many loci of small effect. This is a difficult question to answer, as the detectability of small effect loci is limited. Traditionally, QTL and GWAS approaches can be used to find important genes underlying traits of interest, but suffer from the aforementioned problem of small effect size loci. This research proposes to resolve this issue and provide deeper insights into the genetic architecture of adaptation. Additionally, conducting this research in the maize\//teosinte system informs future research on this important crop and breeding of the species for adapting to future climate change as well as advances our understanding of plant biology and plant genomics. The results could inform % use the word inform too much
crop studies in terms of maintaining genetic diversity for the future in ways that may have much larger long term impacts on the maintenance of diversity as well as on adaptation to changing environments.
% could also mention advantage over human data because have "before and after" data for demographic and selective events 

\paragraph{Broader Impacts}

The impacts of this research will span across both the scientific community and the broader public. The research components and results will be made publicly available  through online repositories of simulation and analysis code as well as public archiving of the data used. Additionally, I will make an effort to publish results from this work either in open-access journals % which is now crazy expensive, but budget could go to this though not top priority
or on public pre-print servers for other journals. I have a strong record of conference attendance that I will maintain in order to present results to the scientific community, and will also use my presence on Twitter and as a contributor to The Molecular Ecologist online blog to more frequently reach both the scientific and public audiences. Furthermore, I will mentor undergraduate students in the lab and also be able to participate in the educational partnership that has been created between UC Davis and Pioneer High School in Woodland, California\kjg{found this here: \url{http://www.dailydemocrat.com/general-news/20140307/uc-davis-partnership-with-pioneer-high-benefits-budding-scientists}}\jri{didnt know about this. would have to get buy-in from brady/sinha. possible if you're interested, but if students want molecular bnechwork experience doesn't fit with current proposal...}. This will allow me to visit high school classrooms to teach about the cutting edge of genomics and evolutionary biology research to young scientists as well as to potentially mentor high school students in the lab or on small-scale projects. \jri{workshop on maize quant gen would be cool. have collaborators at Langebio in Irapuato where you could present. also could design a youtube video on it? another possibility -- i teach undergraduate genetics, and am in process of flipping classroom. you could help design modules (blog posts, videos, etc.) on pop and quant gen for that class, which would potentially could used for all the genetics students at UCD}.% kind of just spitballing right now, need to make sure these things make sense/would work

