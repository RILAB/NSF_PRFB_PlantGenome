%%%%%%%%% PROJECT DESCRIPTION  -- 15 pages (including Prior NSF Support)

\required{Project Description}
\begin{center}
%\emph{Maximum of 15 pages}
\end{center}
%The Project Description (including Results from Prior NSF Support, which is
%limited to five pages) may not exceed 15 pages. Visual materials, including charts,
%graphs, maps, photographs and other pictorial presentations are included in the
%15-page limitation. PIs be cautioned that the project description must
%be self-contained and that URLs that provide information related to the proposal
%should not be used. \\
%
%All proposals to NSF are reviewed utilizing the two merit review criteria,
%intellectual merit and broader impacts. \\
%
% The Project Description should provide a clear statement of the work 
% to be undertaken and must include: objectives for the period of the proposed 
% work and expected significance; relation to longer-term goals of the PI's 
% project; and relation to the present state of knowledge in the field, 
% to work in progress by the PI under other support and to work in progress 
% elsewhere.

\begin{comment}
Prepare Project Description (Research and Training Plan). [6 page limit, including all figures, tables,
etc.]
Select GO next to Project Description and upload file. The bibliography is saved under References Cited
and does not count toward the six--?page limit.
The research and training plan presents the research that you will conduct and the training that you will
receive during the fellowship period and how they relate to your career goals. Include in the research
and training plan:
a) a brief and informative introduction or background section;
b) a statement of research objectives, methods, and significance;
c) training objectives and plan for achieving them (these may include scientific as well as other
career preparation activities);
d) an explanation of how the fellowship activities will enhance your career development and future
research directions as well as describing how this research differs from your dissertation
research, thus providing you an opportunity to broaden your scientific horizon;
e) a justification of the choice of sponsoring scientist(s) and host institution(s);
f) a timetable with yearly goals with benchmarks for major anticipated outcomes; and
g) a separate section within the narrative that describes in detail the broader impacts of the
proposed activities.
\end{comment}

%%%%%%%%%%%%%%%%%%%%%%%%%%%%%%%%%%%%%%%%%%%%%%%%%%%%%%%%%%%%%%%%%%%%%%
%INTRO
%%%%%%%%%%%%%%%%%%%%%%%%%%%%%%%%%%%%%%%%%%%%%%%%%%%%%%%%%%%%%%%%%%%%%%

\section*{A. Introduction}


talk about demographic history broadly and the difficulties it causes

talk about genetic architecture broadly and how important it is to understand for adaptation etc 

then go into how we expect the combination of demog and seln to interact and change genetic architecture in different ways

we cam look at this in maize as it makes a great system to compare two closely related species before and after different demog processes. it also has vast amounts of genomic data available allowing us to investigate the genetic architecture of these pops


%%%%%%%%%%%%%%%%%%%%%%%%%%%%%%%%%%%%%%%%%%%%%%%%%%%%%%%%%%%%%%%%%%%%%%
%OBJECTIVES
%%%%%%%%%%%%%%%%%%%%%%%%%%%%%%%%%%%%%%%%%%%%%%%%%%%%%%%%%%%%%%%%%%%%%%

\section*{B. Research Objectives, Methods \& Significance}
\subsection*{Objective I: do this cool thing}
talk about the objectives here, then list them specifically:

\begin{enumerate}
\item \emph{Do ...?}
\item \emph{Do ...?}
\item \emph{Is there evidence of ...?}
\end{enumerate}

\subsection*{Objective II: Determine the extent to which ...}
Maize was domesticated in...


\subsection*{Rationale and Significance} 
 rationale and significance


%%%%%%%%%%%%%%%%%%%%%%%%%%%%%%%%%%%%%%%%%%%%%%%%%%%%%%%%%%%%%%%%%%%%%%
%TRAINING OBJECTIVES
%%%%%%%%%%%%%%%%%%%%%%%%%%%%%%%%%%%%%%%%%%%%%%%%%%%%%%%%%%%%%%%%%%%%%%
\section*{C. Training Objectives}

%%%%%%%%%%%%%%%%%%%%%%%%%%%%%%%%%%%%%%%%%%%%%%%%%%%%%%%%%%%%%%%%%%%%%%
%CAREER DEVELOPMENT
%%%%%%%%%%%%%%%%%%%%%%%%%%%%%%%%%%%%%%%%%%%%%%%%%%%%%%%%%%%%%%%%%%%%%%
\section*{D. Career Development \& Future Research}



%%%%%%%%%%%%%%%%%%%%%%%%%%%%%%%%%%%%%%%%%%%%%%%%%%%%%%%%%%%%%%%%%%%%%%
% HOST INSTITUTION
%%%%%%%%%%%%%%%%%%%%%%%%%%%%%%%%%%%%%%%%%%%%%%%%%%%%%%%%%%%%%%%%%%%%%%
\section*{E. Sponsoring Scientists and Host Institution}

The University of California Davis (UCD) is the ideal place to conduct the proposed research. UCD has a world-renowned program in evolutionary biology and faculty in population genetics who are at the top of the field. Jeff Ross\--Ibarra is an expert on teosinte, maize, its domestication, and the associated population genetics and genomics of the system. \kjg{Jeff, feel free to make that sound better} 
Kevin Thornton is an accomplished quantitative geneticist and computational biologist who will also contribute greatly to this research. \kjg{Likewise Kevin, feel free to modify}
They will both serve as effective and capable mentors for my post-doctoral research.

in evolutionary biology and plant genetics and many available resources. Graham Coop, Jeff Ross?Ibarra, and Julin Maloof are experts in my proposed research fields and will be capable, effective mentors. In particular, Graham Coop�s expertise in developing methods for detecting polygenic adaptation, Jeff Ross?Ibarra�s knowledge of domestication and experience with the maize?teosinte system and Julin Maloof�s understanding of both regulatory network evolution and the tomato system will be invaluable for my research. Second, the Department of Ecology and Evolution, the Department of Plant Biology, and the Department of Plant Sciences at UCD have many world?class faculty doing research relevant to my interests, so I will have many research groups to interact with. For example, I look forward to interacting with scientists interested in adaptation, such as Johanna Schmitt and Daniel Kliebenstein, and experts in plant regulatory evolution, such as Neelima Sinha, Siobhan Brady, and Daniel Runcie. In addition, there are many resources available at UCD, such as the extensive greenhouse and growth chambers, computing resources, and the UCD Genome Center. Finally, UCD has a vibrant scientific environment which I look forward to participating in.

%%%%%%%%%%%%%%%%%%%%%%%%%%%%%%%%%%%%%%%%%%%%%%%%%%%%%%%%%%%%%%%%%%%%%%
%Milestones
%%%%%%%%%%%%%%%%%%%%%%%%%%%%%%%%%%%%%%%%%%%%%%%%%%%%%%%%%%%%%%%%%%%%%%
\section*{F. Milestones \& Timeline}
\begin{tabular}{ll}
Year 1 \hspace{0.5in} & accomplish this\\
Year 2                     & Finish stuff  \\
\end{tabular}

%%%%%%%%%%%%%%%%%%%%%%%%%%%%%%%%%%%%%%%%%%%%%%%%%%%%%%%%%%%%%%%%%%%%%%
%BROADER IMPACTS
% a separate section within the narrative that describes in detail the broader impacts of the proposed activities.
%%%%%%%%%%%%%%%%%%%%%%%%%%%%%%%%%%%%%%%%%%%%%%%%%%%%%%%%%%%%%%%%%%%%%%
\section*{G. Broader Impacts}





