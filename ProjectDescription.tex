%%%%%%%%% PROJECT DESCRIPTION  -- 15 pages (including Prior NSF Support)

\required{Project Description}
%\begin{center}
%\emph{Maximum of 6 pages}
%\end{center}
%The Project Description (including Results from Prior NSF Support, which is limited to five pages) may not exceed 15 pages. Visual materials, including charts, graphs, maps, photographs and other pictorial presentations are included in the 15-page limitation. PIs be cautioned that the project description must be self-contained and that URLs that provide information related to the proposal should not be used. 

%All proposals to NSF are reviewed utilizing the two merit review criteria, intellectual merit and broader impacts. 

% The Project Description should provide a clear statement of the work to be undertaken and must include: objectives for the period of the proposed  work and expected significance; relation to longer-term goals of the PI's project; and relation to the present state of knowledge in the field, to work in progress by the PI under other support and to work in progress elsewhere.



%%%%%%%%%%%%%%%%%%%%%%%%%%%%%%%%%%%%%%%%%%%%%%%%%%%%%%%%%%%%%%%%%%%%%%
%INTRO
% a brief and informative introduction or background section
%%%%%%%%%%%%%%%%%%%%%%%%%%%%%%%%%%%%%%%%%%%%%%%%%%%%%%%%%%%%%%%%%%%%%%

\section*{A. Introduction}	% might be a little long currently, should not repeat the summary page

A critical goal of evolutionary biology today is to understand organisms' abilities to adapt to new or changing conditions. This is especially relevant in the face of global climate change and other increasingly common anthropogenic changes to the environment \citep{Easterling:2000ja}. Many important phenotypic traits are quantitative (need a ref?), and in order to understand adaptation of such traits, we must understand the genetic architecture underlying them and how this architecture may be changed as a result of selection and demography. The architectures considered in the proposed research concern the number of underlying loci and their various effect sizes and dominance relationships. This informs understanding of the process of adaptation can further benefit crucial goals such as improving crop yields under changing climates.
\jri{in addition we want to know whether standing variation is maintained by selection (possibly adaptive) or at mutation/selection balance (load)}  
% wasn't sure how to fit this ^^ in

\jri{many things effect architecture. num. genes, pop size and demographic history, strength/stability of selection. give examples: human demography (akey), domestication rice/sunflowers (renaut, torsten), RALE (cite kevin2003 \url{http://bioinformatics.oxfordjournals.org/content/19/17/2325.short})} 


% talk about demographic history broadly and the difficulties it causes
The demographic history of populations or species plays a major role in the strength of selection on traits and organisms (cite), and thus has the ability to leave lasting changes or signatures in the genome (cite). A large body of work aims to infer past demographic histories, such as bottlenecks, repeated founder effects during expansion, or rapid population growth, of populations based on signatures in the genome (cite). However, inferring demographic history is a difficult task, as there are many possible histories that may have occurred, and they have the potential to leave similar, or even identical, signals in some species (cite\kjg{e.g. stationary pop growth looks same as range expansion unless you look at the right parameters}). Knowing such demographic history, however, is key, as it has a direct relationship with population size. For example, population bottlenecks greatly decrease the effective population size ($N_e$) which can increase the effect of random genetic drift. The effect size, or strength of selection ($s$), for a given locus, can also impact the efficiency of selection across the genome (selection effective when $N_es>1$). Thus, as demography changes, so does $N_es$ and therefore so does the occurrence of different beneficial, deleterious, or neutral loci across the genome.

% talk about genetic architecture broadly and how important it is to understand for adaptation etc 
The complexity of demographic history is increased manyfold when considering the various architectures that species' genomes comprise. Many ecological traits important for local adaptation have a complex, quantitative genetic basis, and much heterogeneity has been found among these traits both within and among species (cite Orr 2001, Slate 2005). Whether this genetic architecture consists of many loci of small effect or few of large effect can therefore play a role in the impact of demography and selection, as well as mutation, on the genome. How the genome may be restructured as a result of various demographic histories and those effects on selection is not well studied. The genetic architecture underlying traits can affect how easily or quickly local adaptation may occur, or how long and difficult that process may be (cite Yeaman?). Such knowledge can therefore greatly contribute to improve breeding and conservation efforts as well as for predicting responses to environmental changes.

% now into the specifics of this research proposal
\jri{traditionally QTL/GWAS. this good for finding genes, but has limitations for understanding process because power, small fx. we propose to resolve this by 1) using popgen to estimate fitness effects of new mutations and thus effects on a fitness-related phenotype 2) simulation-based approach to estimate genetic architecture of teosinte 3) w/ architecture and DFE, simulate known demography of domestication to see if demography alone explains maize, or demography + selection}

My proposed research aims to investigate these important factors (demographic history, selection, and genetic architecture) and their effects and interactions in terms of shaping the architecture of quantitative traits in maize.  Maize (\emph{Zea mays}, ssp. \emph{mays}) was domesticated from its ancestor \jri{plant genome program was founded by national corn board. we can safely skip basic details of maize}, the wild teosinte, approximately 9,000 years ago in southwestern Mexico (cite Matsuoka et al 2002). In particular, subspecies \emph{Zea mays parviglumis} is the direct progenitor of maize. While two other subspecies of wild teosinte, \emph{mexicana} and \emph{huehuetenangensis}, exist and are found at higher elevations on the Mexican Central Plateau and in western Guatemala, respectively (cite Hufford 2012). Maize and teosinte are extensively studied and serve as ideal models for this research. Demographic history of the species is well documented: supported by both archaeological and other human records of its use and domestication in the Americas (cite). Extensive genetic and genomic data is also available, allowing the known demographic history of the species to be assessed in terms of genomic signatures of domestication (cite?), as well as specifically parameterized in terms of important values such as effective population sizes (cite Beissinger in prep). This provides a unique opportunity to compare these closely related subspecies which have undergone various different demographic processes contributing to their present-day genetic make-up.

In this research project I propose three objectives which make use of the extensive knowledge and data available in the maize\//teosinte system in order to answer the question of \textbf{how the genetic architecture of traits changes as a result of demography and selection}. Objective one uses existing genomic data in teosinte to characterize the genetic architecture of phenotypically important traits. Using the results of this study for parameterizing key genomic characteristics, objective two then simulates the demographic history of teosinte's domestication into modern-day maize. Objective three then characterizes the genetic architecture of the same traits using genomic data from modern maize and compares to theoretical expectations of the genetic architecture of these traits based on the results produced from objective two's simulations. This research will further our understanding of demography and selection's impact on the genome and its architecture, as well as our ability to detect and predict changes in genetic architecture after certain demographic events.



%it is thought, and shown in some human pops, that demog. history such as expansion leads to an increase in delet alleles, and of larger effects - b/c of continued inferred expansions and bottlenecks. 
%is there any evidence of this in maize?
	



%%%%%%%%%%%%%%%%%%%%%%%%%%%%%%%%%%%%%%%%%%%%%%%%%%%%%%%%%%%%%%%%%%%%%%
%OBJECTIVES
% a statement of research objectives, methods, and significance
%%%%%%%%%%%%%%%%%%%%%%%%%%%%%%%%%%%%%%%%%%%%%%%%%%%%%%%%%%%%%%%%%%%%%%

\section*{B. Research Objectives, Methods \& Significance}
\subsection*{Objective I: Investigate the genetic architecture of important traits in teosinte}
\jri{i might make this two objectives. 1) DFE + architecture of teosinte 2) explain architecture of maize}
This first objective aims to estimate the distribution of fitness effects (DFE) in teosinte (\emph{Zea mays parviglumis}), the wild progenitor of domestic maize. The DFE describes the fitness effects of various mutations that are possible within the genome. Mutations can broadly be classified as beneficial, deleterious, or neutral, but in actuality, mutation effect sizes occur on a continuum of strongly deleterious to strongly beneficial, with any value in between. Characterizing the DFE of teosinte uncovers whether the genetic architecture of important quantitative phenotypic traits in teosinte are underlain by many loci of small effect, few loci of large effect, or any combination therein.

Estimating the DFE in teosinte will contribute not only to the purposes of this study, but inform a larger body of work aimed at understanding how common different genetic architectures are for traits important in adaptation. The DFE is difficult to estimate and not broadly understood in evolutionary biology for any given organism or in terms of how much it may vary across organisms. This difficulty arises particularly for cases where there are many loci of small effects: such small effect sizes are difficult to detect on an individual basis. This has implications for studies attempting to identify loci important in adaptation struggle when each locus out of many may only contribute a small amount to the trait (cite Outi Savolainen's work in Scots pine). Thus, studies that aim to identify loci important in adaptation will benefit from knowledge on the range of mutation effect sizes they may expect to see when performing genotype-environment associations or genome-wise association studies (GWAS). 

With the vast genomic resources available in maize, an accurate DFE should be estimable. Using 70 individuals sequenced from a population of teosinte \jri{we estimate DFE from the 70 teos. each sequenced to ~25X. additional data in comments}
%as well as 5000 individuals with genotyping by sequencing (GBS) data\kjg{I might have this data description wrong Jeff? I had in my notes also 4 additional pops in the GWAS study? I think I probably need to add a little more detail on the datasets too}, 

%%%RARE ALLELES PROJECT (probably the most useful data)
% 1 pop of teo w/ 70 parents (sequenced), 5000 progeny (16 phenotypes, GBS, imputation in progress)
% 1 pop of landrace maize w/ 55 parents (sequencing in progress), 5000 progeny (~25 (?) phenotypes, GBS, imputation in progress)
% 4 additional teo pops,  each with 10 parents (sequencing in progress) and 1200 progeny (GBS done, phenotyping in progress done in Jan 2017)
% 4 additional landrace maize pops,  each with 10 parents (sequencing in progress) and 1200 progeny (GBS done, phenotyping in progress done in Jan 2017)

%%OTHER DATA OF USE
% for teosinte architecture
% Weber 2009 teosinte population (if DNAs still available; may be too small sample size pop) 

%for architecture of traits in maize
% NAM!! 5000 RILs, GBS, 42 phenotypes    26 parents crossed to an additional line, 15 million snps

%(suboptimal) alternative DFE, diversity data
% ~1500 sequnced maize genomes (Hapmap 3 and 4) 

%for comparison of maize and teo architecture
% teosinte synthetic: ~2500 DH lines, ~12% teosinte, 40% B73, 2% of each other 25 NAM parents there is dna and tissue
% Briggs 2007 mapping population (600 BC2S3 to estimate ), maybe genotyped on ~1500 markers
% Sherry parviglumis NILs (~800??) there are seeds

% for looking at other landraces
% ~6 landrace genomes each from highland mexico, lowland mexico, highland S. america, lowland s. america, highland guatemala, US southwest
% SeeDs: 
	% 4,000 landraces each w/ 1M SNPs and several phenos (publicly available)
	% 25,000 landraces, each w/ pooled seq. of ~30 plants. + some phenos (not yet avaible, but we know the woman in charge of the program)
	
% other goodies
% 0.2cM-scale recombination map
% rho-map (starting Feb. 2016)

\jri{DoFE should be good for software. we will need to call SNPs and annotate syn/non too} the distribution of fitness effects can be estimated using the DoFE software [cite Fay, Wycoff and Wu (2001), Smith and Eyre-Walker (2002), Bierne and Eyre-Walker (2004) and Eyre-Walker and Keightley (2009), Stoletzki and Eyre-Walker (2011)] which runs a variety of analyses to estimate the components of the distribution of fitness effects. We can apply this methods to sequence data from regions of the genome known to underly important phenotypic traits. Some such quantitative phenotypes include yield, plant height, and flowering time, which are of critical importance to agriculture (cite).

\jri{yes i think GERP for partitioning variance, not sure it's useful as validation of DoFE, will think.} Additionally, it will be possible to validate some of the performance of DoFE on the portion of loci that are inferred to be deleterious. Additional approaches such as GERP or Provean scores (cite) aim to classify the degree of effect size for deleterious mutations, as well as approaches that partition variance components of quantitative traits.
	
\subsection*{Objective II: Simulate the demographic history of maize domestication from teosinte}

\jri{part we need K-Thor for: turning s from DoFE into effect size for quant. trait} This portion of the proposed project aims to use the DFE results of Objective 1 to parameterize a model that will simulate the evolution of maize and its genetic architecture through time during and since its domestication.
% known demog history of maize
Maize originated approximately 9,000 years ago in southern Mexico (cite Matsuoka et al 2002, others?) during a single domestication event of \emph{ssp. parviglumis}. Archaeological records also confirm this dating and single location (cite) as well as the subsequent spread and growth in population size of domestic landraces across the Americas into both lowland and highland environments [cite Wilkes, H. G. (1967) Teosinte: The Closest Relative of Maize (Harvard Univ., Cambridge, MA)]. South American landraces of maize also underwent a second bottleneck event during their expansion (cite). Using genetic data, the precise demographic parameters of this history have been estimated (cite Beissinger et al in prep). This provides information on the ancestral effective population size of maize ($N_a \approx$ 120,000), the size to which the population was bottlenecked during domestication (5\% of this $N_a$), the subsequent size to which populations rapidly expanded (3 times as large as $N_a$\kjg{but maybe as much as 1E9, citation}), and lastly on the genome-wide mutation rate ($3.8\times10^-8$ cite Clark et al 2005 MBE 22, 2304 -- 2312.)

% simulate it with fwdpy 
\jri{will want to think about how to make the comparison.}
The information described above can be used to parameterize simulations of maize genomes undergoing this same demographic history. The DFE of teosinte established from objective 1 provides the necessary parameters for the number of loci and their various effect sizes to be modeled as contributing to each phenotypic trait of interest. Population sizes can be matched to those known during the domestication process, enforcing the desired demographic history, and mutations can be induced at the inferred rates to create simulations as accurate to reality as possible. \kjg{have/need info on recombination/linkage?} The forward simulation program fwdpy (a Python implementation of fwdpp, cite) provides the framework for performing these simulations of complex demographies with natural selection. This approach allows explicit modeling of the genomic architecture desired: distribution of effect sizes across the desired number of loci, including deleterious mutations and their effects on the phenotype. Recombination parameters, as well as setting population sizes, and even sampling from simulated populations is possible.

\jri{this is cool idea. we will need to estimate demography. can use dadi and cite Takuno et al. for basic model. or try MSMC and cite Beissinger? this could be cool 3rd objective, looking at how geography affects. not well done in human lit, but good popgen (as you're aware) describing how this should effect deleterious variants} I will simulate the various landraces of maize which each have a different demographic history during their spread across Central and South America into both lowland and highland environments.\kjg{I think this could be interesting, even if we can't compare them all in the end to real data since we only have it for some - unless it might be in the scope of this project to try and generate some of that data}\jri{have to think about data here. SeeDs is 4K samples but from across Americas. probably not large enough n for GWAS or variance approaches w/in each local region. maybe meixcan highland has enough samples?}  I can then compare the different genetic architectures that evolve from these various conditions. From the simulation results, I will evaluate how many loci are found to contribute to phenotypically important traits (on average and how variable this number is), how strong the mutational effects are at each locus and if and how this relates to the number of loci contributing. This will allow assessment of how important the details of demography are in determining the genetic architecture of local adaptation to different (lowland vs. highland) or similar (lowland Mexico vs. lowland S. America) conditions.\kjg{does this make any sense/is it a valid comparison?} \jri{do we use use teo architecture + demography to sim. pheno and ask if different to infer selection? maybe sketchy? what about Qx-style approach where we simulate demography plus DFE plus architecture in teosinte? how does Qx work if traits is under stabilizing selection in pop1 and there's funky demography in pop2?}

\kjg{goes in intellectual merit section?}
Objective 2 will inform whether the DFE changes under certain demographies, and if so whether there is any meaningful direction in this change. We may hypothesize changes to a wider or narrower or more skewed distribution of fitness effects depending on the demography at hand. Population bottlenecks are a common demographic occurrence during the geographic spread of populations (cite) which will be included in our simulations. Such bottlenecks reduce population sizes greatly, and this reduction can have several effects on the genome including purging of deleterious alleles (recessive alleles become homozygous and are more efficiently removed), or alternatively may lead to an increase of some deleterious alleles through increased random genetic drift (allele surfing, cite Klopfstein et al 2006). The effects of these processes varies depending on the degree of population size reduction and the length of time over which populations are bottlenecked (cite bottleneck lit). The second bottleneck of S. American populations may thus result in a difference between landraces in terms of genetic architecture of adaptive traits. 

\jri{this paragraph great for intellectual merit} These hypothesized changes in the DFE are not well understood in any system and are controversial in humans (cite lohmueller vs. pritchard etc.). Deleterious alleles likely play a large role in many adaptive phenotypes: crop plants have undergone dramatic demographic shifts, usually involving a domestication bottleneck followed by expansion as cultivation spread, and some authors even argue that selection on domestication traits has inadvertently increased the frequency of alleles deleterious for other phenotypes (cite gunther2010). Consistent with this, it has recently been shown that genes associated with a number of quantitative traits in maize are enriched for deleterious alleles compared to randomly chosen genes (cite mezmouk2014). Such information is crucial for understanding variation in phenotype, designing breeding strategies, utilizing diversity from wild relatives, or even engineering new traits using biotechnology. 

% use the same traits from obj. 1, need to describe them there more?
% because then in this section, want to simulate traits w/ varying correlation with fitness
% will need to explain rare alleles pops and experiment some (PGRP15 grant)



\subsection*{Objective III: Estimate and compare the genetic architecture underlying traits in maize post-domestication from simulations and genomic data} 

With the DFE of teosinte estimated in Objective 1 and the simulated evolution of this DFE from teosinte into domesticated maize in Objective 2, we now are primed to compare the predictions of Objective 2 with real world data from modern maize. Similar to Objective 1's approach, Objective 3 will estimate the DFE of modern maize (\emph{Zea mays mays}). The expectation is that if the estimated demographic model and DFE are reasonable, the genetic architecture of simulated phenotypes should closely mimic that of real data found here. If these results are not recapitulated in the maize data, then this indicates that there are complexities in the demographic history and\//or our genomic model parameters that are not well enough understood currently. In this case, we can explore the sensitivity of genetic architecture to changes to the demography or the DFE. Understanding this sensitivity will then lead to improved estimation of these important parameters for the future.

We will recapitulate the methods described in Objective 1 using genomic data from 55 sequenced individuals from a maize population as well as 2,500 individuals from the GWAS datasets\kjg{I definitely just need to chat to you more about the data Jeff, realizing I didn't take good enough notes on things}.

	
	
\jri{broader impacts: simulation pipeline for arbitrary DFE, architecture, selection. claim method to fit to data (do we make that part of the proposal)? info useful for breeders -- learn impact of breeding strategies on traits? }	
	
	
	
	
	
	
%%%%%%%%%%%%%%%%%%%%%%%%%%%%%%%%%%%%%%%%%%%%%%%%%%%%%%%%%%%%%%%%%%%%%%
%TRAINING OBJECTIVES
% training objectives and plan for achieving them (these may include scientific as well as other career preparation activities)
%%%%%%%%%%%%%%%%%%%%%%%%%%%%%%%%%%%%%%%%%%%%%%%%%%%%%%%%%%%%%%%%%%%%%%
\section*{C. Training Objectives}

This research fellowship will provide me with an ideal opportunity to learn the skills I need to enhance my research ability in the fields of genomics and computational biology, both areas in which I expect to continue my future research and which are greatly expanding in evolutionary biology. I will gain many skills related to genomic data analysis through Objectives 1 and 3, learning bioinformatics and analysis skills for large datasets. I have minimal exposure and experience directly working with such data from my dissertation, thus making this a vital step in my career. Genomic technology and data are growing at an incredibly fast pace, and working directly with such data will teach me the most up to date, accurate, and efficient approaches. I will also improve my skills of computational biology through the proposed simulations in Objective 2 and be able to learn a new and useful programming language used widely in evolutionary biology, Python. 


%%%%%%%%%%%%%%%%%%%%%%%%%%%%%%%%%%%%%%%%%%%%%%%%%%%%%%%%%%%%%%%%%%%%%%
%CAREER DEVELOPMENT
% an explanation of how the fellowship activities will enhance your career development and future research directions as well as describing how this research differs from your dissertation research, thus providing you an opportunity to broaden your scientific horizon
%%%%%%%%%%%%%%%%%%%%%%%%%%%%%%%%%%%%%%%%%%%%%%%%%%%%%%%%%%%%%%%%%%%%%%
\section*{D. Career Development \& Future Research}

My career goal is to become an independent, academic researcher who is able to push the boundaries of population genetics and evolutionary biology. As I have in my dissertation research, I aim to study population genetic and genomic processes both empirically from real world data and theoretically through simulations. \jri{careful, you don't want to make this sound like an extension of your PhD} I believe one of the strongest ways to advance our knowledge is through such comparisons of situations where every parameter, current and historical, is known, as well as the evolutionary outcome (simulations) to natural situations where processes or effects that are poorly understood or still unknown to us can diverge from theoretical expectations and provide the basis for further study and investigation into these processes. \jri{somewhere here maybe talk about how with increasing data, pop and quant gen are becoming same? }

The skills I will develop during this fellowship, as described in section C, will benefit my career and put me on the cutting edge for analyses of the newest genomic data and the most recent computational approaches for biological simulations. Interacting with Dr. Ross\--Ibarra, as well as other researchers at UC Davis, and with Dr. Kevin Thornton at UC Irvine, on a regular basis will be both intellectually stimulating and rewarding experiences that will help me accomplish my career goals. Drs. Ross\--Ibarra and Thornton are both at the forefront of a popular \kjg{think of better word?} movement for open science, making all stages of the research process transparent to any interested parties, and providing products such as data and code immediately and publicly \jri{i like the open science angle} . This is a work ethic I strongly agree with and hope to contribute to as an independent researcher. Our work together will better equip me with the tools and experience that make open science easy, efficient, and profitable for all. I believe that this will equip me as a competitive, knowledgeable, and independent researcher able to conduct interesting and useful research throughout my future research program on topics of local adaptation, demographic history, population structure and genetic architecture of important traits.

\jri{if you are considering alternative careers, i'm rather proud of fact i have good track record helping pdocs get jobs: 3 assistant profs, 2 in seed industry, 1 in NGO, 1 govmt. research scientist}
\kjg{yes, I definitely like this, in my head am also considering non-academic careers but worry about coming off as undecided in the proposal?}


%%%%%%%%%%%%%%%%%%%%%%%%%%%%%%%%%%%%%%%%%%%%%%%%%%%%%%%%%%%%%%%%%%%%%%
% HOST INSTITUTION
% a justification of the choice of sponsoring scientist(s) and host institution(s)
%%%%%%%%%%%%%%%%%%%%%%%%%%%%%%%%%%%%%%%%%%%%%%%%%%%%%%%%%%%%%%%%%%%%%%
\section*{E. Sponsoring Scientists and Host Institution}

The University of California Davis (UCD) is the ideal place to conduct the proposed research. UCD has a world-renowned program in evolutionary biology and faculty in population genetics who are at the top of the field. Jeff Ross\--Ibarra is an expert on teosinte, maize, its domestication, and the associated population genetics and genomics of the system. \kjg{Jeff, feel free to make that sound better} 
Kevin Thornton is an accomplished quantitative geneticist and computational biologist at UC Irvine, who will also contribute greatly to this research. \kjg{Likewise Kevin, feel free to modify}
They will both serve as effective and capable mentors for my post-doctoral research. In particular, Jeff has been studying the maize\//teosinte system for many % maybe being vague is okay, or you want to remove entirely? \emph{XX}\kjg{don't forget to fill in}\jri{only 6! i think we sholdn't cite since that doesn't sound like a lot} 
years with a great network of collaborators providing vast resources of data. His work has contributed largely to our knowledge of this system, and more generally on domestication and adaptation as evolutionary processes. Kevin is also the developer and maintainer of fwdpy, the python package proposed for completing the simulations in Objective 2. He will thus serve as a great resource in terms of knowing the exact capabilities of the simulation method and any assumptions of its model that must be taken into account.
Furthermore, the Department of Ecology and Evolution, the Department of Plant Biology, and the Department of Plant Sciences at UCD have many exceptional faculty doing research relevant to my interests, providing many research groups to interact with on a daily basis for potential collaborations or feedback on this research. For example, I look forward to interacting with scientists interested in population genetics, such as Graham Coop, and in adaptation, such as Johanna Schmitt. %Additionally, experts in plant regulatory evolution, such as Neelima Sinha, Siobhan Brady, and Daniel Runcie will serve as great people to interact with. % need to look up people in the dept, these names were just taken from Emily J's application
UCD has the necessary computing resources for our proposed work, and as described, vast sources of knowledge and experience on the topics I plan to investigate, ensuring the success of this work. I am excited to join and contribute to UCD's active and vibrant scientific community. \jri{should probably add something about Farm and Kevin's cluster}


%%%%%%%%%%%%%%%%%%%%%%%%%%%%%%%%%%%%%%%%%%%%%%%%%%%%%%%%%%%%%%%%%%%%%%
%Milestones
% a timetable with yearly goals with benchmarks for major anticipated outcomes
%%%%%%%%%%%%%%%%%%%%%%%%%%%%%%%%%%%%%%%%%%%%%%%%%%%%%%%%%%%%%%%%%%%%%%
\section*{F. Milestones \& Timeline}
\begin{tabular}{ll}
Year 1 \hspace{0.5in} & Estimate DFE in teosinte, get simulations working and compare to teosinte \\
Year 2                     & finish comparison to teosinte, simulate maize \\
Year 3		& expand to other datasets with demography + local adaptation + expansion? \\
\end{tabular}\jri{definitely we go for year 3!}
\kjg{want to do the DFE of maize and teosinte at the same time? no reason we can't right, and then do the simulations. Though if it takes time, teosinte should be prioritized so the sims can be parameterized and started while the maize DFE is estimated} \jri{yup, should be easy to do both at same time.} 

%%%%%%%%%%%%%%%%%%%%%%%%%%%%%%%%%%%%%%%%%%%%%%%%%%%%%%%%%%%%%%%%%%%%%%
%BROADER IMPACTS
% a separate section within the narrative that describes in detail the broader impacts of the proposed activities.
%%%%%%%%%%%%%%%%%%%%%%%%%%%%%%%%%%%%%%%%%%%%%%%%%%%%%%%%%%%%%%%%%%%%%%
\section*{G. Broader Impacts}

The proposed research will have wide-ranging impacts for both the public and the scientific community. I will ensure that my results are available to the public at all stages of these projects by maintaining code and scripts online at my GitHub account, which will allow other researchers to access analysis methods or data cleaning tools as well as simulation details and parameters which can provide a building block from which further research can be conducted. I will present new findings at international conferences and submit publications to open-access pre-print servers. I will also be able to broadcast my work more widely to the public through a strong online presence I maintain on Twitter, blog posts I can contribute to \href{http://www.molecularecologist.com/}{The Molecular Ecologist}, a blog I have contributed to in the past. The wide-ranging impacts of this research on corn as a crop species useful for both food and fuel resources is also of an innate broad impact, as understanding the genetics underlying adaptation will ensure a viable future for such crop species in the future. Lastly, this research will contribute greatly to my own career development, improving my knowledge on genomics and working in an economically important crop species. I will be able to learn these vital tools as well as to teach them to undergraduate students in the lab and into the future as the field of genomics continues to grow.


genetic architecture underlying traits affects how easy/hard and quick/slow local adaptation can occur. important for breeding, conservation, predicting response to climate change, etc.
useful info for crops/domesticated species (and also things like disease in humans?) \jri{yeah, we might think about what advantages we have over humans and how human research can benefit from plants!}


\begin{comment}
Quantitative phenotypes such as yield, plant height, and flowering time are of critical importance to agriculture.  
Deleterious alleles likely play a large role in many of these phenotypes: crop plants have undergone dramatic demographic shifts, usually involving a domestication bottleneck followed by expansion as cultivation spread, and some authors even argue that selection on domestication traits has inadvertently increased the frequency of alleles deleterious for other phenotypes (cite gunther2010). 
Consistent with this hypothesis, my lab has recently shown that genes associated with a number of quantitative traits in maize are enriched for deleterious alleles  compared to randomly chosen genes (cite mezmouk2014).
However, while we know that demography impacts the frequency of individual deleterious variants, we have a poor understanding of the interaction of demography and selection on phenotypic variation. 
In particular, we know little about how these two forces interact to determine the genetic architecture -- the number of genes and their effect -- of a trait. 
Such information is crucial for understanding variation in phenotype, designing breeding strategies, utilizing diversity from wild relatives, or even engineering new traits using biotechnology. 
\end{comment}




