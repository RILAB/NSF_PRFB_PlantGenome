%%%%%%%%% PROJECT DESCRIPTION  -- 15 pages (including Prior NSF Support)

\required{Project Description}
\begin{center}
%\emph{Maximum of 15 pages}
\end{center}
%The Project Description (including Results from Prior NSF Support, which is
%limited to five pages) may not exceed 15 pages. Visual materials, including charts,
%graphs, maps, photographs and other pictorial presentations are included in the
%15-page limitation. PIs be cautioned that the project description must
%be self-contained and that URLs that provide information related to the proposal
%should not be used. \\
%
%All proposals to NSF are reviewed utilizing the two merit review criteria,
%intellectual merit and broader impacts. \\
%
% The Project Description should provide a clear statement of the work 
% to be undertaken and must include: objectives for the period of the proposed 
% work and expected significance; relation to longer-term goals of the PI's 
% project; and relation to the present state of knowledge in the field, 
% to work in progress by the PI under other support and to work in progress 
% elsewhere.

\begin{comment}
Prepare Project Description (Research and Training Plan). [6 page limit, including all figures, tables,
etc.]
Select GO next to Project Description and upload file. The bibliography is saved under References Cited
and does not count toward the six--?page limit.
The research and training plan presents the research that you will conduct and the training that you will
receive during the fellowship period and how they relate to your career goals. Include in the research
and training plan:
a) a brief and informative introduction or background section;
b) a statement of research objectives, methods, and significance;
c) training objectives and plan for achieving them (these may include scientific as well as other
career preparation activities);
d) an explanation of how the fellowship activities will enhance your career development and future
research directions as well as describing how this research differs from your dissertation
research, thus providing you an opportunity to broaden your scientific horizon;
e) a justification of the choice of sponsoring scientist(s) and host institution(s);
f) a timetable with yearly goals with benchmarks for major anticipated outcomes; and
g) a separate section within the narrative that describes in detail the broader impacts of the
proposed activities.
h) For competitive area 1, you must include a plan to broaden participation of under--?represented
groups in biology.
\end{comment}

%%%%%%%%%%%%%%%%%%%%%%%%%%%%%%%%%%%%%%%%%%%%%%%%%%%%%%%%%%%%%%%%%%%%%%
%INTRO
%%%%%%%%%%%%%%%%%%%%%%%%%%%%%%%%%%%%%%%%%%%%%%%%%%%%%%%%%%%%%%%%%%%%%%

\section*{Introduction}
 nicely written intro goes here
 
%%%%%%%%%%%%%%%%%%%%%%%%%%%%%%%%%%%%%%%%%%%%%%%%%%%%%%%%%%%%%%%%%%%%%%
%OBJECTIVES
%%%%%%%%%%%%%%%%%%%%%%%%%%%%%%%%%%%%%%%%%%%%%%%%%%%%%%%%%%%%%%%%%%%%%%

\section*{Objectives}
\subsection*{Objective I: do this cool thing}
talk about the objectives here, then list them specifically:

\begin{enumerate}
\item \emph{Do ...?}
\item \emph{Do ...?}
\item \emph{Is there evidence of ...?}
\end{enumerate}

\subsection*{Objective II: Determine the extent to which ...}
Maize was domesticated in...


\section*{Rationale and Significance} 
 rationale and significance

%%%%%%%%%%%%%%%%%%%%%%%%%%%%%%%%%%%%%%%%%%%%%%%%%%%%%%%%%%%%%%%%%%%%%%
%SPECIFIC OBJECTIVES
%%%%%%%%%%%%%%%%%%%%%%%%%%%%%%%%%%%%%%%%%%%%%%%%%%%%%%%%%%%%%%%%%%%%%%
\section*{Research Plan}


\subsection*{Assess the evolutionary role of ...}
first subsection stuff

\subsubsection*{Does the potential for ...?}
lots of subsection stuff

\subsubsection*{Is i...?}

subsection stuff

\subsubsection*{Can a widespread species serve as ...?}

subsection stuff

\paragraph{Potential Challenges} 
potential challenges 

\section*{Broader Impacts}
stuff here

\subsection*{subsection}
stuff

\subsection*{subsection}
stuff

\required{Results From Prior NSF Support} 


\subsection*{don't think I need this section} 

