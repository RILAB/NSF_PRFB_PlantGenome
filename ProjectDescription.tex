%%%%%%%%% PROJECT DESCRIPTION  -- 15 pages (including Prior NSF Support)

\required{Project Description}
\begin{center}
%\emph{Maximum of 15 pages}
\end{center}
%The Project Description (including Results from Prior NSF Support, which is
%limited to five pages) may not exceed 15 pages. Visual materials, including charts,
%graphs, maps, photographs and other pictorial presentations are included in the
%15-page limitation. PIs be cautioned that the project description must
%be self-contained and that URLs that provide information related to the proposal
%should not be used. \\
%
%All proposals to NSF are reviewed utilizing the two merit review criteria,
%intellectual merit and broader impacts. \\
%
% The Project Description should provide a clear statement of the work 
% to be undertaken and must include: objectives for the period of the proposed 
% work and expected significance; relation to longer-term goals of the PI's 
% project; and relation to the present state of knowledge in the field, 
% to work in progress by the PI under other support and to work in progress 
% elsewhere.

\begin{comment}
Prepare Project Description (Research and Training Plan). [6 page limit, including all figures, tables,
etc.]
Select GO next to Project Description and upload file. The bibliography is saved under References Cited
and does not count toward the six--?page limit.
The research and training plan presents the research that you will conduct and the training that you will
receive during the fellowship period and how they relate to your career goals. Include in the research
and training plan:
a) a brief and informative introduction or background section;
b) a statement of research objectives, methods, and significance;
c) training objectives and plan for achieving them (these may include scientific as well as other
career preparation activities);
d) an explanation of how the fellowship activities will enhance your career development and future
research directions as well as describing how this research differs from your dissertation
research, thus providing you an opportunity to broaden your scientific horizon;
e) a justification of the choice of sponsoring scientist(s) and host institution(s);
f) a timetable with yearly goals with benchmarks for major anticipated outcomes; and
g) a separate section within the narrative that describes in detail the broader impacts of the
proposed activities.
\end{comment}

%%%%%%%%%%%%%%%%%%%%%%%%%%%%%%%%%%%%%%%%%%%%%%%%%%%%%%%%%%%%%%%%%%%%%%
%INTRO
%%%%%%%%%%%%%%%%%%%%%%%%%%%%%%%%%%%%%%%%%%%%%%%%%%%%%%%%%%%%%%%%%%%%%%

\section*{A. Introduction}


talk about demographic history broadly and the difficulties it causes

talk about genetic architecture broadly and how important it is to understand for adaptation etc 

then go into how we expect the combination of demog and seln to interact and change genetic architecture in different ways

we cam look at this in maize as it makes a great system to compare two closely related species before and after different demog processes. it also has vast amounts of genomic data available allowing us to investigate the genetic architecture of these pops


%%%%%%%%%%%%%%%%%%%%%%%%%%%%%%%%%%%%%%%%%%%%%%%%%%%%%%%%%%%%%%%%%%%%%%
%OBJECTIVES
%%%%%%%%%%%%%%%%%%%%%%%%%%%%%%%%%%%%%%%%%%%%%%%%%%%%%%%%%%%%%%%%%%%%%%

\section*{B. Research Objectives, Methods \& Significance}
\subsection*{Objective I: do this cool thing}
talk about the objectives here, then list them specifically:

\begin{enumerate}
\item \emph{Do ...?}
\item \emph{Do ...?}
\item \emph{Is there evidence of ...?}
\end{enumerate}

\subsection*{Objective II: Determine the extent to which ...}
Maize was domesticated in...


\subsection*{Rationale and Significance} 
 rationale and significance


%%%%%%%%%%%%%%%%%%%%%%%%%%%%%%%%%%%%%%%%%%%%%%%%%%%%%%%%%%%%%%%%%%%%%%
%TRAINING OBJECTIVES
%%%%%%%%%%%%%%%%%%%%%%%%%%%%%%%%%%%%%%%%%%%%%%%%%%%%%%%%%%%%%%%%%%%%%%
\section*{C. Training Objectives}

%%%%%%%%%%%%%%%%%%%%%%%%%%%%%%%%%%%%%%%%%%%%%%%%%%%%%%%%%%%%%%%%%%%%%%
%CAREER DEVELOPMENT
%%%%%%%%%%%%%%%%%%%%%%%%%%%%%%%%%%%%%%%%%%%%%%%%%%%%%%%%%%%%%%%%%%%%%%
\section*{D. Career Development \& Future Research}

My career goal is to become an independent, academic researcher who is able to push the boundaries of population genetics and evolutionary biology. As I have in my dissertation research, I aim to study population genetic and genomic processes both empirically from real work data and theoretically through simulations. I believe one of the strongest ways to advance our knowledge is through such comparisons of situations where every parameter, current and historical, is known, as well as the evolutionary outcome (simulations) to natural situations where processes or effects that are poorly understood or still unknown to us can diverge from theoretical expectations and provide the basis for further study and investigation into these processes.

The skills I will develop during this fellowship will benefit my career and put me on the cutting edge for analyses of the newest genomic data and the most recent computational approaches for biological simulations. First, I will gain many skills related to genomic data analysis throughout the course of this research. Genomic technology and data are growing at an incredibly fast pace, and working directly with such data will teach me the most up to date, accurate, and efficient approaches. I will also improve my skills of computational biology through the proposed simulations and be able to learn a new and useful programming language used widely in evolutionary biology, Python. Drs. Ross\--Ibarra and Thornton are both at the forefront of a popular \kjg{think of better word?} movement for open science, making all stages of the research process transparent to any interested parties, and providing products such as data and code immediately and publicly. This is a work ethic I strongly agree with and hope to contribute to as an independent researcher. Our work together will better equip me with the tools and experience that make open science easy, efficient, and profitable for all. Interacting with Drs. Ross\--Ibarra and Thornton, as well as other researchers at UC Davis, on a regular basis will be both intellectually stimulating and rewarding experiences that will help me accomplish my career goals.


%%%%%%%%%%%%%%%%%%%%%%%%%%%%%%%%%%%%%%%%%%%%%%%%%%%%%%%%%%%%%%%%%%%%%%
% HOST INSTITUTION
%%%%%%%%%%%%%%%%%%%%%%%%%%%%%%%%%%%%%%%%%%%%%%%%%%%%%%%%%%%%%%%%%%%%%%
\section*{E. Sponsoring Scientists and Host Institution}

The University of California Davis (UCD) is the ideal place to conduct the proposed research. UCD has a world-renowned program in evolutionary biology and faculty in population genetics who are at the top of the field. Jeff Ross\--Ibarra is an expert on teosinte, maize, its domestication, and the associated population genetics and genomics of the system. \kjg{Jeff, feel free to make that sound better} 
Kevin Thornton is an accomplished quantitative geneticist and computational biologist who will also contribute greatly to this research. \kjg{Likewise Kevin, feel free to modify}
They will both serve as effective and capable mentors for my post-doctoral research. In particular, Jeff has been studying the maize\//teosinte system for \emph{XX}\kjg{don't forget to fill in} years with a great network of collaborators providing vast resources of data. His work has contributed largely to our knowledge of this system, and more generally on domestication and adaptation as evolutionary processes. Kevin is also the developer and maintainer of fwdpy, the python package proposed for completing the simulations in Objective 2. He will thus serve as a great resource in terms of knowing the exact capabilities of the simulation method and any assumptions of its model that must be taken into account.
Furthermore, the Department of Ecology and Evolution, the Department of Plant Biology, and the Department of Plant Sciences at UCD have many exceptional faculty doing research relevant to my interests, providing many research groups to interact with on a daily basis for potential collaborations or feedback on this research. For example, I look forward to interacting with scientists interested in population genetics, such as Graham Coop, and in adaptation, such as Johanna Schmitt. %Additionally, experts in plant regulatory evolution, such as Neelima Sinha, Siobhan Brady, and Daniel Runcie will serve as great people to interact with. % need to look up people in the dept, these names were just taken from Emily J's application
UCD has the necessary computing resources for our proposed work, and as described, vast sources of knowledge and experience on the topics I plan to investigate, ensuring the success of this work. I am excited to join and contribute to UCD's active and vibrant scientific community.


%%%%%%%%%%%%%%%%%%%%%%%%%%%%%%%%%%%%%%%%%%%%%%%%%%%%%%%%%%%%%%%%%%%%%%
%Milestones
%%%%%%%%%%%%%%%%%%%%%%%%%%%%%%%%%%%%%%%%%%%%%%%%%%%%%%%%%%%%%%%%%%%%%%
\section*{F. Milestones \& Timeline}
\begin{tabular}{ll}
Year 1 \hspace{0.5in} & accomplish this\\
Year 2                     & Finish stuff  \\
\end{tabular}
\kjg{want to do the DFE of maize and teosinte at the same time? no reason we can't right, and then do the simulations. Though if it takes time, teosinte should be prioritized so the sims can be parameterized and started while the maize DFE is estimated}

%%%%%%%%%%%%%%%%%%%%%%%%%%%%%%%%%%%%%%%%%%%%%%%%%%%%%%%%%%%%%%%%%%%%%%
%BROADER IMPACTS
% a separate section within the narrative that describes in detail the broader impacts of the proposed activities.
%%%%%%%%%%%%%%%%%%%%%%%%%%%%%%%%%%%%%%%%%%%%%%%%%%%%%%%%%%%%%%%%%%%%%%
\section*{G. Broader Impacts}





