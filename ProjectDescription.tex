%%%%%%%%% PROJECT DESCRIPTION  -- 15 pages (including Prior NSF Support)

\required{Project Description}
%\begin{center}
%\emph{Maximum of 6 pages}
%\end{center}
%The Project Description (including Results from Prior NSF Support, which is limited to five pages) may not exceed 15 pages. Visual materials, including charts, graphs, maps, photographs and other pictorial presentations are included in the 15-page limitation. PIs be cautioned that the project description must be self-contained and that URLs that provide information related to the proposal should not be used. 

%All proposals to NSF are reviewed utilizing the two merit review criteria, intellectual merit and broader impacts. 

% The Project Description should provide a clear statement of the work to be undertaken and must include: objectives for the period of the proposed  work and expected significance; relation to longer-term goals of the PI's project; and relation to the present state of knowledge in the field, to work in progress by the PI under other support and to work in progress elsewhere.



%%%%%%%%%%%%%%%%%%%%%%%%%%%%%%%%%%%%%%%%%%%%%%%%%%%%%%%%%%%%%%%%%%%%%%
%INTRO
% a brief and informative introduction or background section
%%%%%%%%%%%%%%%%%%%%%%%%%%%%%%%%%%%%%%%%%%%%%%%%%%%%%%%%%%%%%%%%%%%%%%

\section*{A. Introduction}	% might be a little long currently, should not repeat the summary page

A critical goal of evolutionary biology today is to understand organisms' abilities to adapt to new or changing conditions. This is especially relevant in the face of global climate change and other increasingly common anthropogenic changes to the environment \citep{Easterling:2000ja}. Many important phenotypic traits are quantitative (need a ref?), and in order to understand adaptation of such traits, we must understand the genetic architecture underlying them and how this architecture may be changed as a result of selection and demography. Such knowledge advances our understanding of the process of adaptation and can further benefit crucial goals such as improving crop yields under changing climates. \jri{in addition we want to know whether standing variation is maintained by selection (possibly adaptive) or at mutation/selection balance (load)}  % wasn't sure how to fit this comment in

Genetic architecture can be affected by many factors, including population size, the number of genes\kjg{in what sense, per trait?}, demographic history, and the strength of positive selection and its consistency through time. Demographic history alone plays a major role in the strength of selection on traits and organisms. For example, human demographic history of changing population sizes, migration, and expansion out of Africa have left lasting signatures in the genome \citep{Fu:2014jt, Gravel:2011iq, Henn:2015dp}. Domestication bottlenecks have also been shown to impact the genome (e.g. in sunflowers \citep{Renaut:2015hi}, torsten - rice?). Such bottlenecks greatly decrease the effective population size ($N_e$) which can increase the effect of random genetic drift. The effect size, or strength of selection ($s$), for a given locus, can also impact the efficiency of selection across the genome (selection effective when $N_es>1$). Thus, as demography changes, so does $N_es$ and therefore so does the occurrence of different beneficial, deleterious, or neutral loci across the genome.

Many traits ecologically important for local adaptation have a complex, quantitative genetic basis, and much heterogeneity has been found among these traits both within and among species \citep{orr:2001, slate:2005}. The architectures considered in the proposed research concern the number of underlying loci and their various effect sizes and dominance relationships. Whether this genetic architecture consists of many loci of small effect or few of large effect can play a role in the impact of demography and selection, as well as mutation, on the genome. How the genome may be restructured as a result of various demographic histories and those effects on selection is not well studied, nor how new genes may evolve to underly traits of importance \citep{Long:2003}. The genetic architecture underlying traits can affect how easily or quickly local adaptation may occur, or how long and difficult that process may be \citep{Yeaman:2015cc, Yeaman:2011jv}\kjg{better refs?}. Such knowledge can therefore greatly contribute to improve breeding and conservation efforts as well as for predicting responses to environmental changes.

My proposed research aims to investigate these important factors (demographic history, selection, and genetic architecture) and their effects and interactions in terms of shaping the architecture of quantitative traits in maize.  Maize (\emph{Zea mays}, ssp. \emph{mays}) was domesticated from its wild teosinte ancestor, \emph{Zea mays}, ssp. \emph{parviglumis}, approximately 9,000 years ago in southwestern Mexico (cite Matsuoka et al 2002). 
% Maize and teosinte are extensively studied and serve as ideal models for this research. 
Demographic history of the species is supported by historic records of its use and domestication in the Americas (cite\kjg{refs?}). Extensive genomic data is also available, allowing the known demographic history of the species to be examined in terms of genomic signatures of domestication, as well as specifically parameterized to match this demographic and selective past in terms of important values such as effective population sizes, or migration rates. This provides a unique opportunity to compare these closely related subspecies which have undergone various different demographic processes contributing to their present-day genetic make-up.

I propose three objectives which make use of the extensive knowledge and data available in the maize\//teosinte system in order to answer the question of \textbf{how the genetic architecture of traits changes as a result of demography and selection}. These approaches overcome issues of similar past studies using GWAS and QTL approaches which suffer from limitations due to inability to detect small effect alleles. Objective one characterizes the genetic architecture of phenotypically important traits in teosinte using existing genomic data to estimate the distribution of effect sizes. Objective two then simulates the demographic history of teosinte's domestication into modern-day maize, parameterized from results of objective one, and compares the expectations created from simulation to the genetic architecture of modern maize. Objective three further investigates the role of demography and selection on changing genetic architecture by simulating this histories of multiple landraces of maize that spread across the Americas post-domestication. This research will further our understanding of demography and selection's impacts on the genome and its structure, as well as our ability to detect and predict changes in genetic architecture after certain demographic events.



%it is thought, and shown in some human pops, that demog. history such as expansion leads to an increase in delet alleles, and of larger effects - b/c of continued inferred expansions and bottlenecks. 
%is there any evidence of this in maize?
	



%%%%%%%%%%%%%%%%%%%%%%%%%%%%%%%%%%%%%%%%%%%%%%%%%%%%%%%%%%%%%%%%%%%%%%
%OBJECTIVES
% a statement of research objectives, methods, and significance
%%%%%%%%%%%%%%%%%%%%%%%%%%%%%%%%%%%%%%%%%%%%%%%%%%%%%%%%%%%%%%%%%%%%%%

\section*{B. Research Objectives, Methods \& Significance}
\subsection*{Objective I: Investigate the genetic architecture of important traits in teosinte}

This first objective aims to estimate the distribution of fitness effects (DFE) in teosinte (\emph{Zea mays} ssp. \emph{parviglumis}), the wild progenitor of domestic maize, and translate this into a distribution of effect sizes. The DFE describes the fitness effects of various mutations that are possible within the genome. Mutations can broadly be classified as beneficial, deleterious, or neutral, but in actuality, mutation effect sizes span a continuum of strongly deleterious to strongly beneficial, with any value in between. The software DoFE \citep{Keightley:2007hq}\kjg{also: Fay, Wycoff and Wu (2001), Smith and Eyre-Walker (2002), Bierne and Eyre-Walker (2004) and Eyre-Walker and Keightley (2009), Stoletzki and Eyre-Walker (2011)} uses of the number of nonsynonymous and synonymous substitutions within genes to calculate the DFE. I will apply this method to sequence data from regions of the genome known to underly important phenotypic traits. Some such quantitative phenotypes include yield, plant height, and flowering time, which are of critical importance to agriculture (cite). From this distribution of fitness effects and fitness measures from these traits, we can then translate these values into effect sizes per allele.\kjg{eek, not sure I'm describing this well} I will use genotypic and phenotypic data on teosinte which is available for 5,000 individuals at 16 phenotypic traits (height, kernel traits...\kjg{do you think I should list them all here?}) (cite data source). These individuals are the progeny of 70 teosinte individuals sequenced to 25X coverage, providing an ideal resource for this analysis.

I will validate these parameter estimates by using them to parameterize simulations of a population of teosinte to an equilibrium state, and by comparing a genome-wide association study (GWAS) on these simulation results to real GWAS results on teosinte, can establish the accuracy of the parameter estimates. If this comparison retuerns vastly different estimates of effect sizes, I will adjust parameter values and resimulate until a sufficient match is made between real teosinte data and simulated data. The simulation program fwdpy (a Python implementation of fwdpp, \citealt{Thornton:2014kn}) allows explicit modeling of the genomic architecture desired: distribution of effect sizes across the desired number of loci, including deleterious mutations and their effects on the phenotype. Values of $V_G$, $V_A$, and $V_D$ can be run to equilibrium, and then the populations sampled in order to test for successful replication of a wild teosinte population.

Characterizing the DFE of teosinte and translating this into a distribution of effect sizes uncovers whether the genetic architecture of important quantitative phenotypic traits in teosinte are underlain by many loci of small effect, few loci of large effect, or any combination therein. Estimating the DFE in teosinte will further inform a larger body of work aimed at understanding how common different genetic architectures are for traits important in adaptation. The DFE is difficult to estimate and not broadly understood in evolutionary biology for any given organism or in terms of how much it may vary across organisms. This difficulty arises particularly for cases where there are many loci of small effects: such small effect sizes are difficult to detect on an individual basis. This has implications for studies attempting to identify loci important in adaptation which struggle when each locus out of many may only contribute a small amount to the trait (cite ? Mackay et al Nature Reviews Genetics 10, 565-577 ?, Rockman 2012 Evolution 61:1�17). Thus, studies that aim to identify loci important in adaptation will benefit from knowledge on the range of mutation effect sizes they may expect to see when performing GWAS or genotype-environment associations. 

%\jri{yes i think GERP for partitioning variance, not sure it's useful as validation of DoFE, will think.} Additionally, it will be possible to validate some of the performance of DoFE on the portion of loci that are inferred to be deleterious. Additional approaches such as GERP or Provean scores (cite) aim to classify the degree of effect size for deleterious mutations, as well as approaches that partition variance components of quantitative traits.
	
	

%_________________% DATA INFORMATION %_________________%
%as well as 5000 individuals with genotyping by sequencing (GBS) data\kjg{I might have this data description wrong Jeff? I had in my notes also 4 additional pops in the GWAS study? I think I probably need to add a little more detail on the datasets too}, 

%%%RARE ALLELES PROJECT (probably the most useful data)
% 1 pop of teo w/ 70 parents (sequenced), 5000 progeny (16 phenotypes, GBS, imputation in progress)
% 1 pop of landrace maize w/ 55 parents (sequencing in progress), 5000 progeny (~25 (?) phenotypes, GBS, imputation in progress)
% 4 additional teo pops,  each with 10 parents (sequencing in progress) and 1200 progeny (GBS done, phenotyping in progress done in Jan 2017)
% 4 additional landrace maize pops,  each with 10 parents (sequencing in progress) and 1200 progeny (GBS done, phenotyping in progress done in Jan 2017)

%%OTHER DATA OF USE
% for teosinte architecture
% Weber 2009 teosinte population (if DNAs still available; may be too small sample size pop) 

%for architecture of traits in maize
% NAM!! 5000 RILs, GBS, 42 phenotypes    26 parents crossed to an additional line, 15 million snps

%(suboptimal) alternative DFE, diversity data
% ~1500 sequnced maize genomes (Hapmap 3 and 4) 

%for comparison of maize and teo architecture
% teosinte synthetic: ~2500 DH lines, ~12% teosinte, 40% B73, 2% of each other 25 NAM parents there is dna and tissue
% Briggs 2007 mapping population (600 BC2S3 to estimate ), maybe genotyped on ~1500 markers
% Sherry parviglumis NILs (~800??) there are seeds

% for looking at other landraces
% ~6 landrace genomes each from highland mexico, lowland mexico, highland S. america, lowland s. america, highland guatemala, US southwest
% SeeDs: 
	% 4,000 landraces each w/ 1M SNPs and several phenos (publicly available)
	% 25,000 landraces, each w/ pooled seq. of ~30 plants. + some phenos (not yet avaible, but we know the woman in charge of the program)
	
% other goodies
% 0.2cM-scale recombination map
% rho-map (starting Feb. 2016)
%___________________________________________________%	
	
\subsection*{Objective II: Simulate the demographic history of maize domestication from teosinte}

The distribution of effect sizes established in objective one are used now to simulate the domestication of maize from teosinte. Again using fwdpy, I will simulate the demographic and selective events that occurred during the domestication of maize. A first step to fit the distribution of effect sizes from objective one will use approximate Bayesian computation (ABC) within fwdpy to estimate the most likely distribution of parameters for $V_G$ and $V_A$.\kjg{did I describe this right?} Simulations will follow the genetic architecture of the aforementioned quantitative traits and their accurate genetic architectures through a strong bottleneck of 5\% of its ancestral $N_e$. Regions of the genome in sizes of 100kb can be simulated with assigned fitness values. Population sizes are easily controlled within fwdpy to mirror the ancestral $N_e$ of maize to its estimated size of $\sim$120,000 followed by the domestication bottleneck which was subsequently followed by a large, rapid growth in population size that expanded populations to at least three times as large as the ancestral $N_e$ (cite). Estimates of the genome-wide mutation rate ($3.8\times10^-8$ \citealt{Clark:2005}) and a 0.2cM-scale recombination map (cite Panzea.org) create simulations that should match reality to the best of our ability. 

I will simulate the same quantitative traits from objective one for which phenotypes and fitness values are known. Interestingly, not all of these traits are purported to be under strong selection during the domestication process. Traits in maize such as kernel weight and kernel row number are expected to have been selected for improving the species as a crop, while there are no strong reasons to suspect other traits such as total plant height were under selection.\kjg{could add in example of tillering and the tbr1 gene?} Therefore, in addition to simulating the demographic bottleneck of the species, I will simulate positive selection on a number of traits assumed to be under artificial selection. 

Having reiterated the origin of modern-day maize in Mexico through simulations, I can now compare the resultant genetic architectures to real genetic architecture in maize data. Genotypic and phenotypic data is currently being generated for another 5,000 progeny individuals of maize, from one landrace of 55 parents (sequenced to 25X) that will be available by the start of this fellowship. Existing data is also available publicly for $\sim$1500 maize genomes through the HapMap 3 and 4 projects (cite) if any unforeseen causes slow the availability of this dataset. The first comparison of results is to see how closely the simulated results match reality. Mismatched results may be attributable to parameter values for selection for which we can then reestimate using ABC, and then resimulate the domestication. Such a result would show that additional traits not expected to be under selection during domestication may actually have experienced more selection than initially assumed. Once closely matched, I can make a second comparison of how the genetic architecture changed over time. I can assess how traits of varying heritabilities have changed over the course of domestication. More heritable traits or traits under positive selection may change in different ways than those more neutrally passing through the bottleneck and I can test if allelic effect sizes have shifted in the distribution's mean or overall shape to more or fewer loci of large effect. 

Furthermore, a validation of the distributions of fitness effects in maize and teosinte can be performed using data from a synthetic cross of maize and teosinte. This \emph{Zea} synthetic results from a mix of 26 maize lines with 12\% teosinte genes (from 11 different founders, fully sequenced), 40\% B73 (the reference genome line), and 2\% from 25 other inbred maize lines. Using this genomic data to again estimate a distribution of effect sizes for the same traits, but now across regions of the genome known to originate from either maize or teosinte, I can test if the respective distributions for maize and teosinte are recovered in the same genetic background. This will identify if our simulations were truly valid.

The results of this project will show the relative importance of demography and selection in determining maize's genetic architecture, and whether information on both processes is necessary to successfully explain the evolution of the architectures underlying important traits. Demographic bottlenecks are common during the geographic spread of populations (cite) and can have several effects on the genome including purging of deleterious alleles (recessive alleles become homozygous and are more efficiently removed), or alternatively may lead to an increase of some deleterious alleles through increased random genetic drift (allele surfing, \citealt{Klopfstein:2005bl}). The effects of these processes varies depending on the degree of population size reduction and the length of time over which populations are bottlenecked (cite bottleneck lit), but also have the potential to interact with the strength of selection during this demographic process. Changes in the distribution of allele effect sizes are not well understood in any system and are controversial in humans \citep{Lohmueller:2014dn, Simons:2014fj, Hancock:2011jb}. Deleterious alleles likely play a large role in many adaptive phenotypes: crop plants have undergone dramatic demographic shifts, usually involving a domestication bottleneck followed by expansion as cultivation spread, and some authors even argue that selection on domestication traits has inadvertently increased the frequency of alleles deleterious for other phenotypes \citep{Gunther:2010}\kjg{is this the right gunther paper?}. Consistent with this, it has recently been shown that genes associated with a number of quantitative traits in maize are enriched for deleterious alleles compared to randomly chosen genes \citep{Mezmouk:2014jd}. Such information is crucial for understanding variation in phenotype, designing breeding strategies, utilizing diversity from wild relatives, or even engineering new traits using biotechnology. 

%Maize originated approximately 9,000 years ago in southern Mexico (cite Matsuoka et al 2002, others?) during a single domestication event of \emph{ssp. parviglumis}. Archaeological records also confirm this dating and single location (cite) as well as the subsequent spread and growth in population size of domestic landraces across the Americas into both lowland and highland environments [cite Wilkes, H. G. (1967) Teosinte: The Closest Relative of Maize (Harvard Univ., Cambridge, MA)]. South American landraces of maize also underwent a second bottleneck event during their expansion (cite). Using genetic data, the precise demographic parameters of this history have been estimated (cite Beissinger et al in prep). This provides information on the ancestral effective population size of maize ($N_a \approx$ 120,000), the size to which the population was bottlenecked during domestication (5\% of this $N_a$), the subsequent size to which populations rapidly expanded (3 times as large as $N_a$\kjg{but maybe as much as 1E9, citation}), and lastly on the genome-wide mutation rate ($3.8\times10^-8$ cite Clark et al 2005 MBE 22, 2304 -- 2312.)


\subsection*{Objective III: Investigate the impact of various demographic and selective pressures on changing genetic of different landraces of maize across the Americas} 

Since domestication, various landraces of maize have spread across Central and South America and adapted into different lowland and highland habitats.\kjg{worth a figure? am short on space already} These populations have experienced further demographic and selective pressures in addition to the initial domestication bottleneck. South American populations are inferred to have experienced a second severe bottleneck (cite), populations expanding geographically are likely to have experienced serial founder effects that can change allele frequencies in unexpected ways due to allele surfing \citep{Klopfstein:2005bl}, and gene flow between teosinte and maize populations may have continued in some cases. Selection pressures in different lowland and highland habitats may also interact with these demographic events.

The objective of this project is to simulate the impact of multiple demographic events and selective pressures either individually or in combination. This will show how much variation is possible in terms of the effects on genetic architecture underlying traits for adaptation. The core set of simulations in this project will be designed to match the known landraces in maize, but will include more broadly a designed experiment to compare the presence or absence of particular evolutionary forces. Though there is currently not sufficient genomic and phenotypic data for all landrace populations of maize, it is likely that this will exist in the future, leaving an excellent future comparison to be made in terms of how well these simulation results match real populations across the species range. The goal of this project is to inform our understanding of the relative importance or insignificance of the tested demographic cases and how sensitive genetic architecture is to these. There is a debate currently in the field of evolutionary biology and human demographic history about the impact of events such as range expansions on the genome and the frequency of deleterious alleles \citep{Henn:2015ce, Sudmant:2015}, necessitating further understanding of how such factors might interact under various selective environments. Furthermore, deleterious alleles likely play a large role in many adaptive phenotypes: crop plants have undergone dramatic demographic shifts, usually involving a domestication bottleneck followed by expansion as cultivation spread, and some authors even argue that selection on domestication traits has inadvertently increased the frequency of alleles deleterious for other phenotypes \citep{Gunther:2010}. Consistent with this, it has recently been shown that genes associated with a number of quantitative traits in maize are enriched for deleterious alleles compared to randomly chosen genes \citep{Mezmouk:2014jd}. Such information is crucial for understanding variation in phenotype, designing breeding strategies, utilizing diversity from wild relatives, or even engineering new traits using biotechnology. 

Similar to objective two, I will simulate regions of the genome within individuals known to underly important phenotypic traits. These individuals will occupy populations that will be subjected to combinations of demographic and selective pressures including some of the following. To recapitulate the various landraces of maize, I will include cases of a second bottleneck after the domestication bottleneck, populations undergoing little or significant additional range expansion (Central American versus South American), stronger or weaker selection on flowering time and phenological traits (warmer lowland versus colder highland adapted populations), and cases with or without gene flow from sympatric\kjg{is that accurate, there is teosinte around them at least?} populations of teosinte. I will do additional simulations to cover a wider range of parameters covering these cases, i.e. more extremes of a longer or larger range expansion, stronger and weaker bottlenecks, and higher or lower levels of gene flow among populations. This will allow assessment of how important the details of demography are in determining the genetic architecture of local adaptation to different conditions.\kjg{feel like I kind of repeated myself at the end here, maybe move the methods in earlier?}

	
%%%%%%%%%%%%%%%%%%%%%%%%%%%%%%%%%%%%%%%%%%%%%%%%%%%%%%%%%%%%%%%%%%%%%%
%TRAINING OBJECTIVES
% training objectives and plan for achieving them (these may include scientific as well as other career preparation activities)
%%%%%%%%%%%%%%%%%%%%%%%%%%%%%%%%%%%%%%%%%%%%%%%%%%%%%%%%%%%%%%%%%%%%%%
\section*{C. Training Objectives}

This fellowship will provide me with an ideal opportunity to learn the skills needed to enhance my ability to conduct cutting edge research in the fields of genomics and computational biology, both areas in which I expect to continue my future research and which are greatly expanding in evolutionary biology. I will gain many skills related to genomic data analysis through these projects, learning bioinformatics and analysis skills of large datasets. I have limited experience working with genomic data from my dissertation, thus making this a vital step in my career. Genomic technology and data are growing at an incredibly fast pace, and working directly with such data will teach me the most up to date, accurate, and efficient approaches. I will also improve my computational biology skills through the proposed simulations, learning a new and useful programming language, Python, that I can apply throughout this research and my future research in evolutionary biology. 


%%%%%%%%%%%%%%%%%%%%%%%%%%%%%%%%%%%%%%%%%%%%%%%%%%%%%%%%%%%%%%%%%%%%%%
%CAREER DEVELOPMENT
% an explanation of how the fellowship activities will enhance your career development and future research directions as well as describing how this research differs from your dissertation research, thus providing you an opportunity to broaden your scientific horizon
%%%%%%%%%%%%%%%%%%%%%%%%%%%%%%%%%%%%%%%%%%%%%%%%%%%%%%%%%%%%%%%%%%%%%%
\section*{D. Career Development \& Future Research}

My career goal is to develop an innovative research program in evolutionary biology, studying population genetics and the processes that impact genetic diversity. I believe such research is key for the future, not only for the field of evolutionary biology, but also in applied scenarios such as understanding prevalence of genetic diseases in humans, adaptation of species to climate change, or strategies for improving agricultural products for a growing world population. My dissertation research has approached some of these questions in a more theoretical and less applied mindset. The work I will conduct during this fellowship would have a much direct potential for application in the field of maize agriculture. For me, this is necessary and vital experience for my career development as I decide between pursuing a more applied research program, potentially in industry or government research scientist positions, or in pursing a career as an academic researcher at a university.

% could maybe still put this in somewhere\jri{somewhere here maybe talk about how with increasing data, pop and quant gen are becoming same? }
The skills I will develop during this fellowship, as described in section C, will benefit my career and put me on the cutting edge for analyses of the newest genomic data and the most recent computational approaches for biological simulations. Interacting with Dr. Ross\--Ibarra, as well as other researchers at UC Davis, and with Dr. Kevin Thornton at UC Irvine, will be both intellectually stimulating and rewarding experiences that will help me accomplish my career goals. Drs. Ross\--Ibarra and Thornton are both at the forefront of a popular movement for open science, making all stages of the research process transparent to any interested parties, and providing products such as data and code immediately and publicly. This is a work ethic I strongly agree with and hope to contribute to as an independent researcher. Our work together will better equip me with the tools and experience that make open science easy, efficient, and profitable for all. I believe that this will equip me as a competitive, knowledgeable, and independent researcher able to conduct interesting and useful research throughout my future research program on topics of local adaptation, demographic history, population structure and genetic architecture of important traits. Furthermore, Dr. Ross-Ibarra has an excellent track record of helping his post-doctoral fellows secure promising positions for their future careers, including 3 assistant professorships at universities, and 4 research scientist positions in the seed industry, an NGO, and the government.\kjg{can I say USDA here Jeff, is this Tim Beissinger?} 


%%%%%%%%%%%%%%%%%%%%%%%%%%%%%%%%%%%%%%%%%%%%%%%%%%%%%%%%%%%%%%%%%%%%%%
% HOST INSTITUTION
% a justification of the choice of sponsoring scientist(s) and host institution(s)
%%%%%%%%%%%%%%%%%%%%%%%%%%%%%%%%%%%%%%%%%%%%%%%%%%%%%%%%%%%%%%%%%%%%%%
\section*{E. Sponsoring Scientists and Host Institution}

The University of California Davis (UCD) is the ideal place to conduct the proposed research. UCD has a world-renowned program in evolutionary biology and faculty in population genetics who are at the top of the field. Jeff Ross\--Ibarra is an expert on teosinte, maize, its domestication, and the associated population genetics and genomics of the system. \kjg{Jeff, feel free to make that sound better} 
Kevin Thornton is an accomplished quantitative geneticist and computational biologist at UC Irvine, who will also contribute greatly to this research. \kjg{Likewise Kevin, feel free to modify}
They will both serve as effective and capable mentors for my post-doctoral research. In particular, Jeff has been studying the maize\//teosinte system for many % maybe being vague is okay, or you want to remove entirely? \emph{XX}\kjg{don't forget to fill in}\jri{only 6! i think we sholdn't cite since that doesn't sound like a lot} 
years with a great network of collaborators providing vast resources of data. His work has contributed largely to our knowledge of this system, and more generally on domestication and adaptation as evolutionary processes. Kevin is also the developer and maintainer of fwdpy, the python package proposed for completing the simulations in Objective 2. He will thus serve as a great resource in terms of knowing the exact capabilities of the simulation method and any assumptions of its model that must be taken into account.
Furthermore, the Department of Ecology and Evolution, the Department of Plant Biology, and the Department of Plant Sciences at UCD have many exceptional faculty doing research relevant to my interests, providing many research groups to interact with on a daily basis for potential collaborations or feedback on this research. For example, I look forward to interacting with scientists interested in population genetics, such as Graham Coop, and in adaptation, such as Johanna Schmitt. %Additionally, experts in plant regulatory evolution, such as Neelima Sinha, Siobhan Brady, and Daniel Runcie will serve as great people to interact with. % need to look up people in the dept, these names were just taken from Emily J's application
UCD has the necessary computing resources for our proposed work, and as described, vast sources of knowledge and experience on the topics I plan to investigate, ensuring the success of this work. I am excited to join and contribute to UCD's active and vibrant scientific community. \jri{should probably add something about Farm and Kevin's cluster}


%%%%%%%%%%%%%%%%%%%%%%%%%%%%%%%%%%%%%%%%%%%%%%%%%%%%%%%%%%%%%%%%%%%%%%
%Milestones
% a timetable with yearly goals with benchmarks for major anticipated outcomes
%%%%%%%%%%%%%%%%%%%%%%%%%%%%%%%%%%%%%%%%%%%%%%%%%%%%%%%%%%%%%%%%%%%%%%
\section*{F. Milestones \& Timeline}
\begin{tabular}{ll}
Year 1 \hspace{0.5in} & Estimate DFE in teosinte, get simulations working and compare to teosinte \\
Year 2                     & finish comparison to teosinte, simulate maize \\
Year 3		& expand to other datasets with demography + local adaptation + expansion? \\
\end{tabular}\jri{definitely we go for year 3!}
\kjg{want to do the DFE of maize and teosinte at the same time? no reason we can't right, and then do the simulations. Though if it takes time, teosinte should be prioritized so the sims can be parameterized and started while the maize DFE is estimated} \jri{yup, should be easy to do both at same time.} 

%%%%%%%%%%%%%%%%%%%%%%%%%%%%%%%%%%%%%%%%%%%%%%%%%%%%%%%%%%%%%%%%%%%%%%
%BROADER IMPACTS
% a separate section within the narrative that describes in detail the broader impacts of the proposed activities.
%%%%%%%%%%%%%%%%%%%%%%%%%%%%%%%%%%%%%%%%%%%%%%%%%%%%%%%%%%%%%%%%%%%%%%
\section*{G. Broader Impacts}

The proposed research will have wide-ranging impacts for both the public and the scientific community. I will ensure that my results are available to the public at all stages of these projects by maintaining code and scripts online at my GitHub account, which will allow other researchers to access analysis methods or data cleaning tools as well as simulation details and parameters which can provide a building block from which further research can be conducted. I will present new findings at international conferences and submit publications to open-access pre-print servers. I will also be able to broadcast my work more widely to the public through a strong online presence I maintain on Twitter, blog posts I can contribute to \href{http://www.molecularecologist.com/}{The Molecular Ecologist}, a blog I have contributed to in the past. The wide-ranging impacts of this research on corn as a crop species useful for both food and fuel resources is also of an innate broad impact, as understanding the genetics underlying adaptation will ensure a viable future for such crop species in the future. Lastly, this research will contribute greatly to my own career development, improving my knowledge on genomics and working in an economically important crop species. I will be able to learn these vital tools as well as to teach them to undergraduate students in the lab and into the future as the field of genomics continues to grow.


genetic architecture underlying traits affects how easy/hard and quick/slow local adaptation can occur. important for breeding, conservation, predicting response to climate change, etc.
useful info for crops/domesticated species (and also things like disease in humans?) \jri{yeah, we might think about what advantages we have over humans and how human research can benefit from plants!}


\begin{comment}
Quantitative phenotypes such as yield, plant height, and flowering time are of critical importance to agriculture.  
Deleterious alleles likely play a large role in many of these phenotypes: crop plants have undergone dramatic demographic shifts, usually involving a domestication bottleneck followed by expansion as cultivation spread, and some authors even argue that selection on domestication traits has inadvertently increased the frequency of alleles deleterious for other phenotypes (cite gunther2010). 
Consistent with this hypothesis, my lab has recently shown that genes associated with a number of quantitative traits in maize are enriched for deleterious alleles  compared to randomly chosen genes (cite mezmouk2014).
However, while we know that demography impacts the frequency of individual deleterious variants, we have a poor understanding of the interaction of demography and selection on phenotypic variation. 
In particular, we know little about how these two forces interact to determine the genetic architecture -- the number of genes and their effect -- of a trait. 
Such information is crucial for understanding variation in phenotype, designing breeding strategies, utilizing diversity from wild relatives, or even engineering new traits using biotechnology. 
\end{comment}




