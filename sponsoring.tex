\renewcommand{\thepage}{Sponsoring Scientist Statement - Page \arabic{page} of 3}
\required{Sponsoring Scientist Statement}

\subsection{Brief description of the research projects in the host research group(s), including a statement of current and pending research support} \jri{Kevin, please add}
%, both private and public, for each sponsor. If any sponsor has submitted similar research for funding, the degree of overlap must be addressed.
Research in the Ross-Ibarra lab focuses on the evolutionary genetics of maize and teosinte, including studies of local adaptation, genome evolution, and experimental evolution. The proposed project is related to the goals of the funded ``Biology of Rare Alleles'' grant.  That grant seeks to understand the impacts of rare alleles and predict phenotypes from genotyping data. While this proposal will ideally utilize data produced by that grant, its focus on the mechanisms of evolutionary change and the proposed modeling framework are outside of the scope of the ``Biology of Rare Alleles'' grant. \jri{is this credible enough, or do I need to say more?}

\subsubsection*{Current Support}
\jri{kevin please add!}
\textbf{Title:} Adaptive gene flow from teosinte to highland maize in central Mexico
\textbf{Agency:} UC MEXUS (Ross-Ibarra, Co-PI)
\textbf{Amount:} \$24,897
\textbf{Dates:} 07/15– 12/16
\textbf{Calendar Person-months:} 0.24

\noindent\textbf{Title:} Functional genomics of maize centromeres 
\textbf{Agency:} NSF-PGRP (Ross-Ibarra, Co-PI)
\textbf{Amount:} \$754,409
\textbf{Dates:} 06/10 – 5/16
\textbf{Calendar Person-months:} 2.4

\noindent\textbf{Title:} US Mexico planning visit and workshop to assess the genomic basis of local adaptation in maize 
\textbf{Agency:} NSF-CNIC (Ross-Ibarra, Co-PI)
\textbf{Amount:} \$36,450
\textbf{Dates:} 09/14 – 12/16
\textbf{Calendar Person-months:} 0.5

\textbf{Title:} Biology of rare alleles in maize and its wild relatives 
\textbf{Agency:} NSF-PGRP (Ross-Ibarra, Co-PI)
\textbf{Amount:} \$3,221,212
\textbf{Dates:} 05/15 – 4/18
\textbf{Calendar Person-months:} 2

\subsubsection*{Pending Support}\\ 
\noindent\textbf{Title:} Evolutionary genomics of maize 
\textbf{Agency:} HHMI (Ross-Ibarra, PI)
\textbf{Amount:} NA (decided by agency)
\textbf{Dates:} 07/16 – 6/21
\textbf{Calendar Person-months:} 1

\noindent\textbf{Title:} Research PGR: The genetics of highland adaptation in maize
\textbf{Agency:} NSF PGRP (Ross-Ibarra, PI)
\textbf{Amount:} \$4,531,773
\textbf{Dates:} 05/16 – 4/21
\textbf{Calendar Person-months:} 2.4

\noindent\textbf{Title:} Research PGR: The evolutionary role of hybridization and introgression in the genus Zea
\textbf{Agency:} NSF-DEB (Ross-Ibarra, Co-PI)
\textbf{Amount:} \$415,775
\textbf{Dates:} 07/16 – 6/19
\textbf{Calendar Person-months:} 0.6

\subsection{Description of how the research and training plan for the applicant would fit into and complement ongoing research of the sponsor(s) as well as an indication of the personnel with whom the Fellow would work.}\jri{Kevin please add/expand}

The proposed research nicely extends current work in the Ross-Ibarra lab on the impacts of demography and selection on allele frequencies to their effects on complex phenotypes. By investigating the genetic architecture may change, this proposal also shows potential to be extremely informative of work underway on complementation and the basis of hybrid vigor.  
Kim will be encouraged to interact with other postdoctoral fellows and graduate students in both labs.  These interactions provide opportunities to work outside the focus of the current proposal, to develop new collaborations, and to coauthor additional publications.  We hope and expect that some of these interactions may evolve into formal collaborations on projects not involving either sponsor.  

\subsection{Explanation of how the sponsor(s) will determine what mentoring the applicant needs in research, teaching, and career development skills and how these would be translated into a specific plan that fosters the development of the applicant's future independent research career.}
\textbf{Research} We have planned a rigorous guided reading discussion to ensure Kim gets quickly up to speed on the necessary quantitative genetics theory as well as the relevant literature on quantitative traits in maize and discussed with Kim online and formal coursework options for learning python programming. We will both meet with Kim weekly to ensure progress on the research; Jeff also has an open-door policy encouraging interaction with lab members outside of formal meeting times. 

\textbf{Writing} We work with lab members to develop an initial outline of their papers, but lab members are expected to prepare a complete first draft of manuscripts and figures. This responsibility is reflected in the fact that postdocs and students are not only first but also co-corresponding author on their papers. We then work with lab members to revise and edit their manuscript in a process that often goes through numerous rounds of revision. Manuscripts are shared with other lab members and colleagues to encourage practice providing (and receiving) constructive criticism, and final versions of our papers are posted as preprints online upon submission to take advantage of community feedback that can be incorporated into the published paper.  One clear goal will be first authorship on submitted papers, with the expectation of approximately one first author paper per year of duration of the postdoc.  We will also encourage lab members to participate in other writing activities, including blog posts on our journal club papers, formal reviews for journals, and grant proposals.  The Ross-Ibarra lab has a documented history of successful funding with postdoctoral scholars as Co-PIs, and these efforts provide valuable training (and even initial funding) for the scholars' future academic careers.  

\textbf{Teaching}  Postdoctoral fellows in the Ross-Ibarra Lab are encouraged to participate in formal teaching opportunities including development of lectures for a graduate course on ecological genomics or a large undergraduate genetics lecture course.  Jeff works with the postdoctoral fellow to develop the lecture and provides feedback on teaching style and and content afterwards. Kim will also be encouraged to practice presenting her work to broad audiences both in joint lab meetings and formal seminars at UC Davis and UC Irvine, as well as at local (e.g. Bay Area Population Genomics) and international (SMBE, PAG) meetings. Both Jeff and Kevin provide guidance and feedback on such presentations. 

\textbf{Mentoring}
Another important aspect of training will be experience mentoring graduate students and undergraduates.  Previous efforts to encourage such supervision in our labs have been very successful, with postdoc-mentored students presenting conference posters on their research or earning authorship on papers.  Supervisory experience has proven helpful for postdocs applying for jobs, especially in industry.

\textbf{Career Development}
During weekly  meetings we will focus not only on current progress on the project but also on progress  towards Kim's broader career goals. We will encourage Kim to take advantage of formal infrastructure for professional development in place at UCDavis, including training in responsible conduct of research, grantsmanship, mentoring, career development, authorship of journal papers, and teaching. We will encourage attendance at conferences to present results and build relationships with other leaders in the field. The Ross-Ibarra lab has a proven record of success in placing postdoctoral scholars in careers industry, government, nongovernmental research organizations, and academic positions, and we will continue to encourage postdocs to explore a range of career opportunities.

\subsection{Description of the role the sponsors will play in the proposed research and training and the other resources that will be available to the Fellow to complete his or her training plan during the fellowship.}\jri{Kevin please add resources etc. at UCI}

Kevin will primarily help Kim to develop the necessary models and code, and Jeff will primarily help in analysis of simulation and empirical data and guidance in the interpretation of the results. 
However, Kim will meet weekly with both advisors to facilitate research interactions among the sponsors and ensure that everyone is in agreement on progress and continued plans.   We will additionally formally address long terms goals and progress in separate meetings every 6 months. Though Kim will be based primarily in the Ross-Ibarra lab at UC Davis, we expect the development of the simulation framework will involve extended visits to the Thornton lab at UC Irvine.

UC Davis is regularly ranked as one of the top three centers of research in both plant and evolutionary biology.  Kim will have access to a broad set of discussion groups and seminars, as well as to the experience and guidance of world-class set of scientists that can help her with all aspects of the development of her research, engagement in diversity issues, and career path.

Both labs have access to extensive computational resources. The Ross-Ibarra lab is a contributing partner in the college  computer cluster, giving the lab dedicated access to 192 processors, with the opportunity for use of nearly 800 additional CPU as resources allow. Recent additions to the cluster have provided it with additional CPU as well as six new shared high-memory (512Gb RAM) nodes, one of which is dedicated to the Ross-Ibarra lab. Dr. Ross-Ibarra is a faculty member of the UC Davis Genome Center, a large facility that includes bioinformatics and genotyping cores as well as access to additional computational facilities. 

\subsection{Description of the limitations, if any, be placed on the Fellow regarding the research following the fellowship.}
Kim would be free to continue and expand on the proposed work either in maize or other systems and to continue development of fwdpy simulation code.
%